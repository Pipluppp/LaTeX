\documentclass{article}
\usepackage{siunitx}

\title{Groups}
\author{Duncan F. Bandojo \\\ 11-Newton}


\makeatletter


\usepackage{amsfonts}
\usepackage{amsmath}
\usepackage{amssymb}
\usepackage{amsthm}
\usepackage{booktabs}
\usepackage{enumitem}
\usepackage{fancyhdr}
\usepackage{mathdots}
\usepackage{mathtools}

\pagestyle{fancy}
\fancyhf{}
\lhead{\leftmark}
\cfoot{\thepage}

%\usepackage{graphicx}
%\usepackage{textcomp}
%\usepackage{slashed}
%\usepackage{tabularx}
%\usepackage[normalem]{ulem}
%\usepackage[all]{xy}
%\usepackage{imakeidx}
%\usepackage{microtype}
%\usepackage{multirow}
%\usepackage{siunitx}
%\usepackage{alltt}
%\usepackage{caption}

% Theorems
\theoremstyle{definition}
\newtheorem*{aim}{Aim}
\newtheorem*{axiom}{Axiom}
\newtheorem*{claim}{Claim}
\newtheorem*{cor}{Corollary}
\newtheorem*{conjecture}{Conjecture}
\newtheorem*{defi}{Definition}
\newtheorem*{eg}{Example}
\newtheorem*{ex}{Exercise}
\newtheorem*{fact}{Fact}
\newtheorem*{law}{Law}
\newtheorem*{lemma}{Lemma}
\newtheorem*{notation}{Notation}
\newtheorem*{prop}{Proposition}
\newtheorem*{question}{Question}
\newtheorem*{problem}{Problem}
\newtheorem*{rrule}{Rule}
\newtheorem*{thm}{Theorem}
\newtheorem*{assumption}{Assumption}
\newtheorem*{result}{Result}

\newtheorem*{remark}{Remark}
\newtheorem*{warning}{Warning}
\newtheorem*{exercise}{Exercise}


%%%%%%%%%%%%%%%%%%%%%%%%%
%%%%% Maths Symbols %%%%%
%%%%%%%%%%%%%%%%%%%%%%%%

% Special sets
\newcommand{\C}{\mathbb{C}}
\newcommand{\CP}{\mathbb{CP}}
\newcommand{\GG}{\mathbb{G}}
\newcommand{\N}{\mathbb{N}}
\newcommand{\Q}{\mathbb{Q}}
\newcommand{\R}{\mathbb{R}}
\newcommand{\RP}{\mathbb{RP}}
\newcommand{\T}{\mathbb{T}}
\newcommand{\Z}{\mathbb{Z}}
\renewcommand{\H}{\mathbb{H}}

% Brackets
\newcommand{\abs}[1]{\left\lvert #1\right\rvert}
\newcommand{\set}[1]{\left\{ #1\right\}}

\let\stdsection\section
\renewcommand\section{\newpage\stdsection}

\makeatother


\begin{document}
\maketitle

\tableofcontents
\setcounter{section}{-1}
\section{Introduction}
This lectures are based on the course Math 4120 (Modern Algebra), which is based
on the book Visual Group Theory. These are rough notes about the video lectures
of the course.

\section{Groups, intuitively}
\subsection{What is a Group?}
Our introduction to group theory will begin by discussing the famous Rubik’s
Cube

Four Key Observations:
\begin{itemize}
        \item There is a predefined list of moves that never changes
        \item Every move is reversable
        \item Every move is deterministic
        \item Moves can be combined in any sequence 
\end{itemize}
Group theory studies the mathematical consequences of these $4$ observations, which
in turn will help us answer interesting questions about symmetrical objects.

\subsubsection{Informal Definition of a Group}
\begin{itemize}
     \item[1.] There is a predefined list of \emph{actions} that never changes
     \item[2.] Every action is reversible
     \item[3.] Every action is deterministic
     \item[4.] Any sequence of consecutive actions is also an action
\end{itemize}
\begin{defi}[Informal]
     A \emph{group} is a set of actions satisfying Rules $1–4$.
\end{defi}

A group is not the set of all possible configurations, but the set of actions.

\subsubsection{Observations about the “Rubik’s Cube group”}
Frequently, two sequences of moves will be “indistinguishable.” We will say that
two such moves are the same. For example, rotating a face (by $\ang{90}$) once
has the same effect as rotating it five times.
\begin{fact}
    There are $43,252,003,274,489,856,000$ distinct configurations of the
    Rubik’s cube
\end{fact}
While there are infinitely many possible sequences of moves, starting from the solved
position, there are $43,252,003,274,489,856,000$ “\emph{truly distinct}" moves.
\medbreak
All $4.3 \times 10^{19}$ moves are \emph{\textbf{generated}} by just $6$ moves: a $\ang{90}$ clockwise
twist of one of the $6$ faces.
\medbreak
Let’s call these generators $a, b, c, d, e,$ and $f$ . Every \emph{word} over the
alphabet $\{a, b, c, d, e, f \}$ describes a unique configuration of the cube
(starting from the solved position).
\medbreak
\subsubsection{Summary of the big ideas}

Loosely speaking a \emph{group} is a "\emph{set of actions}" satisfying some
mild properties: deterministic, reversibility, and closure.

\begin{defi}[Generating Set]
    A \emph{generating set} for a group is a subcollection of actions that together
    can produce all actions in the group – like a spanning set in a vector space.
\end{defi}

Usually, a generating set is much smaller than the whole group.
\medbreak
Given a generating set, the individual actions are called \emph{generators}.
\medbreak
The set of all possible ways to scramble a Rubik’s cube is an example of a
group. Two actions are the same if they have the same “net effect”, e.g.,
twisting a face $1$ time vs. twisting a face $5$ times.
\medbreak
Note that the group is the set of actions one can perform, not the set of
configurations of the cube. However, there is a bijection between these two
sets.
\medbreak
The Rubik’s cube group has $4.3 \times 10^{19}$ actions but we can find a
generating set of size $6$.

\subsection{Cayley Graphs}
\subsubsection{The Rectangle Puzzle Group}
Let $G$ denote the rectangle group. This is a set of four actions. We see:
\begin{itemize}
    \item $G$ has $4$ actions: the "identity" action $e$, a horizontal flip $h$,
    a vertical flip $v$, and a $\ang{180}$ rotation $r$.
    \[G = \{e,h,v,r\}\].
    \item We need two actions to “generate” $G$. In our diagram, each
    \emph{generator} is represented by a different type (color) of arrow. We
    write:
    \[G = \langle h,v\rangle\]
    \item The map shows us how to get from any one configuration to any other.
    There is more than one way to follow the arrows! For example
    \[r = hv= vh\].
    \item For this particular group, the order of the actions is irrelevant! We
    call such a group \emph{abelian}. Note that the Rubik’s cube group is not abelian.
    \item Every action in $G$ is its own inverse: That is
    \[e=e^2=h^2=v^2=r^2\].
    \item The Rubik’s cube group does not have this property. Algebraically, we
    write:
    \[e^{-1}=e, \qquad v^{-1}=v, \qquad h^{-1}=h, \qquad r^{-1}=r\].
\end{itemize}
The rectangle puzzle can also generated by a \emph{horizontal flip} and a $\ang{180}$
rotation
\[G = \langle h,v\rangle\]

\subsection{Cayley Diagrams}
In general, a Cayley diagram consists of \emph{nodes} that are connected by
colored arrows, where:
\begin{itemize}
     \item an arrow of a particular color  represents a specific \emph{generator}.
     \item each action of the group is represented by a unique \emph{node} (sometimes
     we will label nodes by the corresponding action).
     \item Equivalently, each \emph{action} is represented by a (non-unique) \emph{path} starting from the
     solved state.
     \item An arrow corresponding to the generator $g$ from node $x$ to node $y$
     means that node $y$ is the result of applying the action $g \in G$ to node
     $x$
     \item 
\end{itemize}
\end{document}