\subsection{Functions}
\begin{defi}[Axioms of Equality]
    a relation linking two objects $x,y$ of the same type $T$. How equlity is
    defined depends on the class $T$ of objects under consideration. We require
    equality to obey the four \emph{axioms of equality}.
    \begin{itemize}
        \item (Reflexive Axiom). Given any object $x$, we have $x=x$.
        \item (Symmetry Axioms). Given any two objects $x$ and $y$ of the same
        type, if $x=y$, then $y=x$.
        \item (Transitive Axiom). Given any three objects $x,y,z$ of the same
        type, if $x=y$ and $y=z$, then $x=z$.
        \item (Substitution Axiom). Given any two objects $x$ and $y$ of the
        same type, if $x=y$, then $f(x)=f(y)$ for all functions or operations
        $f$.
    \end{itemize}
\end{defi}
\begin{defi}[Equality of Functions] Functions $f:X\to Y$ and $g:X\to Y$ with
same domain and range are equal, if and only if $f(x)=g(x)$ for all $x\in X$.
     
\end{defi}
\begin{exercise}[3.3.1] 
     Show that the definition of the equality of functions is a
     reflexive, symmetric, and transitive. Also verify the substitution
     proeprty: if $f, \tilde{f}:X \to Y$ and $g,\tilde{g}:Y\to Z$ are functions
     such that $f=\tilde{f}$ and $g=\tilde{g}$, then $g\circ f = g\circ
     \tilde{f}$.
     \begin{proof}
          We seperate each property and make use of the \emph{Axioms of
          Equality}.
          \begin{itemize}
              \item(Reflexive). Let $f:X\to Y$ be a function and $x\in X$. Then
              there exists $f(x)\in Y$, and by the \emph{reflexive axiom} of
              equality
              \begin{equation*}
                  f(x)=f(x).
              \end{equation*}
              Hence, $f=f$, since $x$ is an arbitrary element of $X$.
              \item(Symmetric). Let $f:X\to Y$and $g:X\to Y$ be functions and
              $x\in X$ be arbitrary. Assume for $f(x), g(x)\in Y$ that
              \begin{equation*}
                  f(x)=g(x),
              \end{equation*}
              by the \emph{symmetry axiom} of equality of objects of the same type,
              \begin{equation*}
                  g(x)=f(x).
              \end{equation*}
              Therefore, $f=g$ implies $g=f$.
              \item(Transitivity). Let $f,g,h$ be functions with domain $X$ and
              range $Y$. Assume that $f=g$ and $g=h$, so that for all $x\in X$,
              \begin{equation*}
                  f(x)=g(x) \quad \text{and} \quad g(x)=h(x),
              \end{equation*}
              where $f(x),g(x),h(x)\in Y$. By the \emph{trasitivity axiom} of
              equality, 
              \begin{equation*}
                  f(x)=h(x).
              \end{equation*}
              Since $x$ is arbitrary, $f=h$.
              \pagebreak
              \item(Substitution). Let $f,\tilde{f}:X\to Y$ and
              $g,\tilde{g}:Y\to Z$ be functions such that $f=\tilde{f}$ and
              $g=\tilde{g}$. Hence, $f(x)=\tilde{f}(x)$ for all $x\in X$ and
              $g(y)=\tilde{g}(y)$ for all $y\in Y$.

              To show that $g\circ f = \tilde{g}\circ \tilde{f}$, we must show
              that $(g\circ f)(x)=(\tilde{g}\circ \tilde{f})(x)$ for all $x\in
              X$. The two functions already have the same domain and range.

              We make use of the \emph{subsitution axiom} of equality, on the
              elements $f(x),\tilde{f}(x)\in Y$, and notice that $g,\tilde{g}$
              are functions.
              \begin{equation*}
                  f(x)=\tilde{f}(x) \quad \Longrightarrow \quad g(f(x))=g(\tilde{f}(x)),
              \end{equation*}
              and it follows that 
              \begin{equation*}
                  g(\tilde{f}(x))=\tilde{g}(\tilde{f}(x)).
              \end{equation*}
              Making use of the \emph{axioms of equality}, 
              \begin{equation*}
                  (g\circ f)(x)=g(f(x))=g(\tilde{f}(x))=\tilde{g}(\tilde{f}(x))=(\tilde{g}\circ \tilde{f})(x).
              \end{equation*}
              Which holds for all $x\in X$, hence, $g\circ f = \tilde{g}\circ \tilde{f}$.
          \end{itemize}
     \end{proof}
\end{exercise}
\begin{exercise}[3.3.2] Let $f:X\to Y$ and $g:Y\to Z$ be functions. Show that if
$f$ and $g$ are both injective, then so is $g\circ f$; similarly, show that if
$f$ and $g$ are both surjective, then so is $g\circ f$.
     \begin{proof} There are two proofs we need to do:
          \begin{itemize}
              \item(Injective). Let $x,x'\in X$ be arbitrary. Assume that $f$ and $g$ are both
              injective. Since $g$ is injective and $f(x),f(x')\in Y$,
              \begin{equation*}
                  g(f(x)) = g(f(x')) \implies f(x)=f(x'). 
              \end{equation*}
              We also know that $f$ is injective and $x,x' \in X$ so that
              \begin{equation*}
                  f(x)=f(x') \implies x=x'.
              \end{equation*}
              Therefore, 
              \begin{equation*}
                  g(f(x))=g(f(x'))\implies x=x',
              \end{equation*}
              Hence, $g\circ f$ is injective.
              \item(Surjective). Assume $f$ and $g$ are surjective. By the
              definition of surjection, 
              \begin{equation*}
                  \forall z\in Z, \exists y\in Y \;st.\; g(y)=z,
              \end{equation*}
              and 
              \begin{equation*}
                  \forall y \in Y \exists x \in X \;st.\; f(x)=y.
              \end{equation*}
              Therefore, for every $z\in Z$, we can choose an $x\in X$ such that
              \begin{equation*}
                  z= g(y) = g(f(x)) = (g\circ f)(x).
              \end{equation*}
              Hence, $g\circ f$ is surjective.
          \end{itemize}
     \end{proof}
\end{exercise}
\pagebreak
\begin{exercise}[3.3.3] When is the \emph{empty function} injective? surjective?
    bijective? (The empty function $f:\emptyset \to Y$ is a function with the
    empty set as its domain).
     \begin{proof} A function $f$ is injective if for arbitrary $x,x'$ in its
     domain,
     \begin{equation*}
         x\neq x' \implies f(x)\neq f(x').
     \end{equation*}
     Notice that there exists no element in the empty set, hence the
     \emph{empty function} is always injective no matter the range is.

     A function is surjective if for every $y$ in its range, there exists $x$
     in its domain such that
     \begin{equation*}
         f(x)=y.
     \end{equation*}
     Notice that if the range is nonempty, then there exists an element in the
     range wherein $f(x)=y$ is not satisfied since the empty set has no
     element. Therefore, if the range is empty, then the \emph{empty function}
     is surjective vacuously.

     The \emph{empty function} is bijective if the range is empty.
    \end{proof}
\end{exercise}
\begin{exercise}[3.3.4] Let $f:X\to Y, \tilde{f}:X\to Y, g:Y\to Z, $and
$\tilde{g}:y\to Z$ be functions. Show that if $g\circ f = g\circ \tilde{f}$ and
$g$ is injective, then $f=\tilde{f}$. Is the same statement true if $g$ is not
injective? Show that if $g\circ f=\tilde{g}\circ f$ and $f$ is injective, then
$g=\tilde{g}$. Is the same statement true if $f$ is not surjective?
     \begin{proof}
          
     \end{proof}
\end{exercise}  