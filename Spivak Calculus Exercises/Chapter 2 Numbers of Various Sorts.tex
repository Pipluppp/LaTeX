\pagebreak
\subsection{Number of Various Sorts}
\begin{defi}[Mathematical Induction] Suppose that $P(x)$ means that the property
$P$ holds for the number $x$. Then the principle of mathematical induction
states that $P(x)$ holds for all natural numbers $x$ provided that
\begin{itemize}
    \item $P(1)$ is true.
    \item Whenever $P(k)$ is true, $P(k+1)$ is true.
\end{itemize}     
\end{defi}

(Note: In the construction of the natural numbers from the \emph{Peano Axioms},
induction is an axiom and $0$ is an element of the naturals.)

\begin{defi}[Strong Induction] We use a stronger hypothesis since in some cases
it happens that in order to prove $P(k+1)$ we must assume not only $P(k)$, but
also $P(l)$ for all natural numbers $l \leq k$. $P(x)$ holds for all natural
numbers $x$ provideded that
\begin{itemize}
    \item $P(1)$ is true
    \item $P(k+1)$ is true, if $P(l)$ is true for $1<l \leq k$
\end{itemize}
\end{defi}

\begin{defi}[Recursive Definitions] A \emph{recursive definition} of a function
defines values of the function for some inputs in terms of the values of the
function for other inputs.
\end{defi}

For example, the number $n!$ is defined as the product of all then natural
numbers less then or equal to $n$:
\begin{equation*}
    n! = 1\ \cdot 2 \cdot 3 \cdot \dotsc \cdot (n-1) \cdot n.
\end{equation*} 
More formally:
\begin{align}
    1! &= 1 \\
    n!&=n\cdot (n-1)!.
\end{align}

Another example is the convenient notation regarding sums. Instead of writing 
\begin{equation*}
    a_1 + a_1 + \dotsb + a_n.
\end{equation*}
we will use the Greek letter $\Sigma$ and write
\begin{equation*}
    \sum \limits_{i=1}^{n} a_i. 
\end{equation*}

The define $\sum \limits_{i=1}^{n} a_i$ precisely really requires a recursive
definition:
\begin{align}
    \sum \limits_{i=1}^{1} a_i &= a_1, \\
    \sum \limits_{i=1}^{n} a_i &= \sum \limits_{i=1}^{n-1} a_i + a_n.   
\end{align}

As an example where induction is used for its proof, 
\begin{equation*}
    \sum \limits_{i=1}^{n} i = \frac{n(n+1)}{2}
\end{equation*}

\pagebreak
\subsubsection{Exercises}
\begin{exercise}[\textbf{1}] Prove the following formulas by induction.
     \begin{enumerate}
         \item $1^2 + \dotsb + n^2 = \dfrac{n(n+1)(2n+1)}{6}$.
         \begin{proof} We induct on $n$. For the base case $n=1$, the equation
         is trivial.
        \begin{align*}
            1^2 &= \frac{1(1+1)(2(1)+1)}{6} \\
            &= \frac{2(3)}{6} \\
            &= 1
        \end{align*}
        Now we assume that the equation hold for arbitrary $n$, and we shall
        prove that it also hold for $n+1$.
        \begin{align*}
            1^2+\dotsb +n^2+(n+1)^2 &= \sum \limits_{i=1} ^{n+1} i^2 \\
            &=  \sum \limits_{i=1} ^{n} i^2 + (n+1)^2 \\
            &= \frac{n(n+1)(2n+1)}{6} + (n+1)^2 \\
            &= \frac{n(n+1)(2n+1)+6(n+1)^2}{6} \\
            &= \frac{(n+1)[2n^2+n+6n+6]}{6} \\
            &= \frac{(n+1)(n+2)(2(n+1)+1)}{6}.
        \end{align*}
        Hence, the equation hold for all natural numbers.
         \end{proof}
         \item $1^3+\dotsb +n^3 =(1+\dotsb +n)^2$.
         \begin{proof} We induct on $n$ and prove the base case $n=1$,
              \begin{equation*}
                  1^3 = (1)^2
              \end{equation*}
              which is trivial. 
              
              Let us now assume that the equation hold for
              arbitrary $n$. We must now show that it also hold for $n+1$.
              \begin{align*}
                  (1+\dotsb +(n+1))^2 &= (n+1)(2(1+\dotsb +n)+(n+1)) + (1+\dotsb +n)^2 \\
                  &= (1^3+\dotsb +n^3)+(n+1)(2(1+\dotsb +n)+(n+1)) \\
                  &= (1^3+\dotsb +n^3) + (n+1)(2\cdot \frac{n(n+1)}{2}+n+1) \\
                  &= (1^3+\dotsb +n^3) + (n+1)((n+1)(n+1)) \\
                  &= 1^3+\dotsb +(n+1)^3,
              \end{align*}
              hence, the equation holds for all natural numbers.
         \end{proof}
     \end{enumerate}
\end{exercise}
\begin{exercise}[\textbf{2}] Find a formula for 
     \begin{enumerate}
        \item $\sum \limits_{i=1} ^{n} (2n-1) = 1+3+5+\dotsb +(2n-1)$.
        \begin{proof}We can make use of the properties of the summation symbol
        and we can easily derive the formla. However, we will not make use of it.
             \begin{align*}
                \sum \limits_{i=1} ^{n} (2n-1) &= 1+3+5+\dotsb +(2n-1) \\
                &= 1+2+3+\dotsb +(2n) - 2(1+2+3+\dotsb +n) \\
                &= \frac{2n(2n+1)}{2} - 2\cdot \frac{n(n+1)}{2} \\
                &= n((2n+1)-(n+1)) \\
                &=n^2.
             \end{align*}
        \end{proof}
        \item $\sum \limits_{i=1} ^{n} (2n-1)^2 = 1^2+3^2+5^2+\dotsb +(2n-1)^2$.
        \begin{proof} We can make use of the properties of the summation symbol
            and we can easily derive the formla. However, we will not make use of it.
             \begin{align*}
                \sum \limits_{i=1} ^{n} (2n-1)^2 &= 1^2+3^2+5^2+\dotsb +(2n-1)^2 \\
                &= 1^2+2^2+3^2+\dotsb +(2n)^2 - 4(1^2+2^2+3^2+\dotsb +n^2) \\
                &= \frac{2n(2n+1)(4n+1)}{6} - 4\cdot \frac{n(n+1)(2n+1)}{6} \\
                &= \frac{2n(2n+1)\left(4n+1-2(n+1)\right)}{6} \\
                &= \frac{n(2n+1)(2n-1)}{3}.
             \end{align*}
        \end{proof}
     \end{enumerate}
\end{exercise}
\pagebreak
\begin{exercise}[\textbf{3}] If $0\leq k \leq n$, the \emph{"binomial
    coefficient"} $\dbinom{n}{k}$ is defined by 
    \begin{equation*}
        \binom{n}{k} = \frac{n!}{k!(n-k)!} = \frac{n(n-1)\dotsb (n-k+1)}{k!}, \text{if }k\neq 0, n 
    \end{equation*}
     and a special case of the first formula if we define $0!=1$,
     \begin{equation*}
         \binom{n}{0}=\binom{n}{n}=1,
     \end{equation*}    
     and for $k<0$ or $k>n$ we just define the binomial coefficient to be $0$.
     \begin{enumerate}
         \item Prove that
         \begin{equation*}
             \binom{n+1}{k}=\binom{n}{k-1}+\binom{n}{k}.
         \end{equation*}
         \begin{proof} It is just a manipulation of the expressions and from the fact that,
            \begin{equation*}
                \binom{n+1}{k} = \frac{(n+1)!}{(n+1-k)!k!}.
            \end{equation*}
            Now for the proof,
              \begin{align*}
                \binom{n}{k-1}+\binom{n}{k} &= \frac{n!}{(n+1-k)!(k-1)!} +\frac{n!}{(n-k)!k!}\\
                  &= \frac{n!k+n!(n+1-k)}{(n+1-k)!k!} \\
                  &= \frac{(n+1)!}{(n+1-k)!k!}.
              \end{align*}
         \end{proof}
         This relation gives rise to the following configuration, known as
         \emph{"Pascal's Triangle"}\textemdash  a number not on one of the sides
         is the sum of the two numbers above it; the binomial coefficient
         $\dbinom{n}{k}$ is the $(k+1)$st number in the $(n+1)$st row.
         \item Notice that all the numbers in Pascal's Triangle are natural
         numbers. Use part (i) to prove by induction that $\dbinom{n}{k}$ is
         always a natural number.
         \begin{proof} We want to prove the assertion that for fixed $n$ (Note: The case $k=0$ is trivial),
            \begin{equation*}
                \binom{n}{k} \text{is a natural number for all }k, 1\leq k \leq n.
            \end{equation*}
            We prove it by induction on $n$. For the base case where $n=1$, we
            only need to prove for $k=1$.
            \begin{equation*}
                \binom{1}{1} =1.
            \end{equation*}
            For the inductive step, we assume that the assertion is true for
            arbitrary $n$ and we will show that 
            \begin{equation*}
                \binom{n+1}{k} \text{ is a natural number for all }k, 1\leq k\leq n+1.
            \end{equation*}
            The case where $k=1$ and $k=n+1$ are trivial. We can now assume that
            for $2\leq k\leq n$, it must be that  $1\leq k-1< k\leq n$ and we
            can make use this with part (i)
            \begin{equation*}
                \binom{n+1}{k} = \binom{n}{k-1}+\binom{n}{k}, 
            \end{equation*}
            since the sum of naturals is a natural, $\binom{n+1}{k}$ is
            natural. We have shown that $\binom{n+1}{k}$ is natural for
            all $k$, $1\leq k \leq n+1$ if $\binom{n}{k}$ is natural for all k
            $1\leq k \leq n$. Hence, the induction is complete.
         \end{proof}
         \item Give another proof that $\dbinom{n}{k}$ is natural number by
         showing that $\dbinom{n}{k}$ is the number of sets of exactly $k$
         integers each chosen from $1,\dotsc , n$.
         \begin{proof} In choosing $k$ objects from $n$, there are $n(n-1)\dotsb
         (n-k+1)$ ways to choose a set with $k$ elements, $\{k_1,\dotsc, k_n
         \}$. However, for each set the $k$ objects can be arranged in $k!$
         ways, that is why we divide by $k!$ as the order of the $k$th element
         in a certain set with the same elements does not matter. Hence, 
         \begin{equation*}
             \binom{n}{k} = \frac{n(n-1)\dotsc (n-k+1)}{k!} = \frac{n!}{(n-k)!k!},
         \end{equation*}
              is the number of sets of exactly $k$ integers chosen from $n$ and
              that is why $\binom{n}{k}$ is always a natural.
         \end{proof}
         \item Prove the \emph{"binomial theorem"}: If $a$ and $b$ are any
         numbers and $n$ is a natural number, then 
         \begin{align*}
             (a+b)^n &= a^n+\binom{n}{1}a^{n-1}b+\binom{n}{2}a^{n-2}b^2 +\dotsb +\binom{n}{n-1}ab^{n-1}+b^n \\
             &= \sum \limits_{j=0}^{n}\binom{n}{j}a^{n-j}b^j.
         \end{align*}
         \begin{proof}
              To be continued...
         \end{proof}
         \end{enumerate}
\end{exercise}
