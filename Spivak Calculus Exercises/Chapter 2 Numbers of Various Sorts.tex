\pagebreak
\subsection{Number of Various Sorts}
\begin{defi}[Mathematical Induction] Suppose that $P(x)$ means that the property
$P$ holds for the number $x$. Then the principle of mathematical induction
states that $P(x)$ holds for all natural numbers $x$ provided that
\begin{itemize}
    \item $P(1)$ is true.
    \item Whenever $P(k)$ is true, $P(k+1)$ is true.
\end{itemize}     
\end{defi}

(Note: In the construction of the natural numbers from the \emph{Peano Axioms},
induction is an axiom and $0$ is an element of the naturals.)

\begin{defi}[Strong Induction] We use a stronger hypothesis since in some cases
it happens that in order to prove $P(k+1)$ we must assume not only $P(k)$, but
also $P(l)$ for all natural numbers $l \leq k$. $P(x)$ holds for all natural
numbers $x$ provideded that
\begin{itemize}
    \item $P(1)$ is true
    \item $P(k+1)$ is true, if $P(l)$ is true for $1<l \leq k$
\end{itemize}
\end{defi}

\begin{defi}[Recursive Definitions] A \emph{recursive definition} of a function
defines values of the function for some inputs in terms of the values of the
function for other inputs.
\end{defi}

For example, the number $n!$ is defined as the product of all then natural
numbers less then or equal to $n$:
\begin{equation*}
    n! = 1\ \cdot 2 \cdot 3 \cdot \dotsc \cdot (n-1) \cdot n.
\end{equation*} 
More formally:
\begin{align}
    1! &= 1 \\
    n!&=n\cdot (n-1)!.
\end{align}

Another example is the convenient notation regarding sums. Instead of writing 
\begin{equation*}
    a_1 + a_1 + \dotsb + a_n.
\end{equation*}
we will use the Greek letter $\Sigma$ and write
\begin{equation*}
    \sum \limits_{i=1}^{n} a_i. 
\end{equation*}

The define $\sum \limits_{i=1}^{n} a_i$ precisely really requires a recursive
definition:
\begin{align}
    \sum \limits_{i=1}^{1} a_i &= a_1, \\
    \sum \limits_{i=1}^{n} a_i &= \sum \limits_{i=1}^{n-1} a_i + a_n.   
\end{align}

As an example where induction is used for its proof, 
\begin{equation*}
    \sum \limits_{i=1}^{n} i = \frac{n(n+1)}{2}
\end{equation*}

\pagebreak
\subsubsection{Exercises}
\begin{exercise}[\textbf{1}] Prove the following formulas by induction.
     \begin{enumerate}
         \item $1^2 + \dotsb + n^2 = \dfrac{n(n+1)(2n+1)}{6}$.
         \begin{proof} We induct on $n$. For the base case $n=1$, the equation
         is trivial.
        \begin{align*}
            1^2 &= \frac{1(1+1)(2(1)+1)}{6} \\
            &= \frac{2(3)}{6} \\
            &= 1
        \end{align*}
        Now we assume that the equation hold for arbitrary $n$, and we shall
        prove that it also hold for $n+1$.
        \begin{align*}
            1^2+\dotsb +n^2+(n+1)^2 &= \sum \limits_{i=1} ^{n+1} i^2 \\
            &=  \sum \limits_{i=1} ^{n} i^2 + (n+1)^2 \\
            &= \frac{n(n+1)(2n+1)}{6} + (n+1)^2 \\
            &= \frac{n(n+1)(2n+1)+6(n+1)^2}{6} \\
            &= \frac{(n+1)[2n^2+n+6n+6]}{6} \\
            &= \frac{(n+1)(n+2)(2(n+1)+1)}{6}.
        \end{align*}
        Hence, the equation hold for all natural numbers.
         \end{proof}
         \item $1^3+\dotsb +n^3 =(1+\dotsb +n)^2$.
         \begin{proof} We induct on $n$ and prove the base case $n=1$,
              \begin{equation*}
                  1^3 = (1)^2
              \end{equation*}
              which is trivial. 
              
              Let us now assume that the equation hold for
              arbitrary $n$. We must now show that it also hold for $n+1$.
              \begin{align*}
                  (1+\dotsb +(n+1))^2 &= (n+1)(2(1+\dotsb +n)+(n+1)) + (1+\dotsb +n)^2 \\
                  &= (1^3+\dotsb +n^3)+(n+1)(2(1+\dotsb +n)+(n+1)) \\
                  &= (1^3+\dotsb +n^3) + (n+1)(2\cdot \frac{n(n+1)}{2}+n+1) \\
                  &= (1^3+\dotsb +n^3) + (n+1)((n+1)(n+1)) \\
                  &= 1^3+\dotsb +(n+1)^3
              \end{align*}
              Hence, the equation holds for all natural numbers.
         \end{proof}
     \end{enumerate}
\end{exercise}