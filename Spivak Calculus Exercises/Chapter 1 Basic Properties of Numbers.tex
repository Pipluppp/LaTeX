\section{Prologue}
\subsection{Numbers of Various Sorts}
\begin{defi}[Field Properties] The following properties hold in $\R$
    
     \begin{itemize}
         \item[P1](Associative law for addition) \hfill $a+(b+c) = (a+b)+c$.
         \item[P2](Existence of an additive identity) \hfill $a+0=0+a=a$.
         \item[P3](Existence of additive inverse) \hfill $a+(-a)=(-a)+a=0 $.
         \item[P4](Commutative law for addition) \hfill $a+b=b+a$.
         \item[P5](Associative law for multiplication) \hfill $a\cdot(b\cdot c)
         = (a\cdot b)\cdot c$.
         \item[P6](Existence of multiplicative identity) \hfill $a\cdot 1 =
         1\cdot a = a;\quad 1 \neq 0$.
         \item[P7](Existence of multiplicative inverses) \hfill $a\cdot a^{-1} =
         a^{-1}\cdot a = 1,\text{ for } a \neq 0$.
         \item[P8](Commutative law for multiplication) \hfill $a\cdot b = b\cdot a$.
         \item[P9](Distributive law) \hfill $a\cdot(b+c) = a\cdot b+ a\cdot c$.
         \item[P10](Trichotomy law) For every number $a$, one and only one of
         the following holds: (Denote $P$ as the collection of positive numbers)
            \begin{enumerate}
                \item $a=0$,
                \item $a $ is in the collection $P$,
                \item $-a $ is in the collection $P$.
            \end{enumerate}
        \item[P11](Closure under addition) If $a $ and $ b $ are in $ P $, then
        $a+b$ is in $P$.
        \item[P12](Closure under multiplication) If $a$ and $b$ are in $P$, then
        $a\cdot b$ is in $P$.
     \end{itemize}
\end{defi}
\begin{thm}[Triangle Inequality]
     For all numbers $a$ and $b$, we have
     \begin{equation*}
         \abs{a+b} \leq \abs{a}+\abs{b}
     \end{equation*}
\end{thm}
\begin{exercise}[\textbf{1}]
     Prove the following:
    \begin{enumerate}
        \item If $ax=a$ for some number $a \neq 0$, then $x=1$.
        \begin{proof}
            Assume that $ax=a$ fro some number $a\neq 0$. 
            \begin{align*}
                x = x\cdot 1 = x\cdot (a\cdot a^{-1}) &= ax \cdot (a^{-1}) \\
                &= a \cdot (a^{-1}) \\
                &= (a\cdot a^{-1}) \\   
                &= 1
            \end{align*}
        \end{proof}
        \item $x^2-y^2 = (x-y)(x+y)$.
        \begin{proof} Using the field axioms.
             \begin{align*}
                 (x-y)(x+y) &= x\cdot (x+y) + (-y)\cdot (x+y) \\
                 &= (x^2+xy)+((-y)\cdot x+(-y)\cdot y) \\
                 &= x^2+xy-xy-y^2 \\
                 &= x^2-y^2
             \end{align*}
        \end{proof}
        \item If $x^2=y^2$, then $x=y$ or $x=-y$.
        \begin{proof}Assume that $x^2=y^2$. We make use of (ii).
             \begin{align*}
                 x^2 = y^2 &\Leftrightarrow  x^2-y^2 = 0 \\
                 & \Leftrightarrow (x-y)(x+y) =0 \\
                 & \Rightarrow x=y \text{ or } x=-y.
             \end{align*}
        \end{proof}
        \item $x^3-y^3=(x-y)(x^2+xy+y^2)$.
        \begin{proof}
             Using the field axioms
             \begin{align*}
                 (x-y)(x^2+xy+y^2) &= x^2(x-y)+xy(x-y)+y^2(x-y) \\
                 &= (x^3-x^2y)+(x^2y-xy^2)+(xy^2-y^3) \\
                 &= x^3+(x^2y-x^2y)+(xy^2-xy^2)-y^3 \\
                 &= x^3-y^3
             \end{align*}
        \end{proof}
        \item $x^n-y^n = (x-y)(x^{n-1}+x^{n-2}y+\dotsb +xy^{n-2}+y^{n-1}).$
        \begin{proof}Using the field axioms
             \begin{align*}
                 (x-y)(x^{n-1}+x^{n-2}y+\dotsb +xy^{n-2}+y^{n-1}) &= x(x^{n-1}+x^{n-2}y+\dotsb+xy^{n-2}+y^{n-1}) \\
                 & \qquad -[y(x^{n-1}+x^{n-2}y+\dotsb+xy^{n-2}+y^{n-1})] \\
                 &= x^n+x^{n-1}y+\dotsb+x^2y^{n-2}+xy^{n-1} \\
                 & \qquad -[x^{n-1}y+x^{n-2}y^2+\dotsb+xy^{n-1}+y^{n}] \\
                 &= x^n-y^n
             \end{align*}
        \end{proof}
        \begin{proof}[Alternative Proof] We make us of sigma notation
             \begin{align*}
                 (x-y)\cdot \sum_{i=0}^{n-1}x^iy^{n-(i+1)} &= x\left(\sum_{i=0}^{n-1}x^iy^{n-(i+1)}\right)- \left[y\left(\sum_{i=0}^{n-1}x^iy^{n-(i+1)}\right)\right] \\
                 &= \sum_{i=0}^{n-1}x^{i+1}y^{n-(i+1)} -\left[\sum_{i=0}^{n-1}x^iy^{n-i}\right] \\
                 &= x^n +\sum_{i=0}^{n-2}x^{i+1}y^{n-(i+1)} -\left[\sum_{i=1}^{n-1}x^iy^{n-i} +y^n\right] \\
                 &= x^n +\sum_{i=0}^{n-2}x^{i+1}y^{n-(i+1)} -\left[\sum_{i=0}^{n-2}x^{i+1}y^{n-(i+1)} +y^n\right] \\
                 &= x^n-y^n + \sum_{i=0}^{n-2} \left[x^{i+1}y^{n-(i+1)} - (x^{i+1}y^{n-(i+1)}) \right] \\
                 &= x^n-y^n +\sum_{i=0}^{n-2} 0 \\
                 &= x^n-y^n
             \end{align*}
        \end{proof}
        \item $x^3+y^3 = (x+y)(x^2-xy+y^2).$
        \begin{proof} Replace $y$ by $-y$ in part (iv)
             \begin{align*}
                x^3-y^3 = (x-y)(x^2+xy+y^2) & \Leftrightarrow x^3-(-y)^3 = (x-(-y))(x^2+x(-y)+(-y)^2) \\
                & \Leftrightarrow x^3+y^3 = (x+y)(x^2-xy+y^2)
             \end{align*}
        \end{proof}
    \end{enumerate}
\end{exercise}

\begin{exercise}[\textbf{2}]
     What is wrong with the following "proof"? Let $x=y$. Then
     \begin{align*}
         x^2 &= xy, \\
         x^2-y^2 &= xy-y^2, \\
         (x+y)(x-y) &= y(x-y), \\
         x+y &= y, \\
         2y &= y, \\ 
         2 &= 1.    
     \end{align*}
     \begin{proof}[Solution] For all $a \in \R$ we know that $a\cdot a^{-1}=0$
     with the assumption $a\neq 0$. The $4th$ step is contradictory on the given
     fact that $x=y$ which implies $x-y=0$ and has no multiplicative inverse.
    \end{proof}
\end{exercise}
\begin{exercise}[\textbf{3}] Prove the following:
     \begin{enumerate}
         \item $\dfrac{a}{b} = \dfrac{ac}{bc}$, if $b,c \neq 0$.
         \begin{proof} Using the field axioms
              \begin{align*}
                  \dfrac{a}{b} = ab^{-1} &= (ab^{-1})(c\cdot c^{-1}) \\
                  &= (ac)(b^{-1}c^{-1}) \\
                  &= (ac)(bc)^{-1} \\
                  &= \dfrac{ac}{bc}
              \end{align*}
         \end{proof}
         \item $\dfrac{a}{b}+\dfrac{c}{d} = \dfrac{ad+bc}{bd}$, if $b,d \neq 0$.
         \begin{proof} Using the field axioms   
              \begin{align*}
                  \dfrac{a}{b}+\dfrac{c}{d} = ab^{-1}+cd^{-1} &= (ab^{-1}+cd^{-1})\cdot (bd)(bd)^{-1} \\
                  &= (ad(b\cdot b^{-1})+bc(d\cdot d^{-1}))\cdot (bd)^{-1} \\
                  &= (ad+bc)\cdot (bd)^{-1} \\
                  &= \dfrac{ad+bc}{bd}
              \end{align*}
         \end{proof}
         \pagebreak
         \item $(ab)^{-1} = a^{-1}b^{-1}$, if $a,b\neq 0$. 
         \begin{proof} Using the field axioms
              \begin{align*}
                  ab(a^{-1}b^{-1}) &=  1 \\
                  a^{-1}b^{-1} &= (ab)^{-1}
              \end{align*}
         \end{proof}
         \item $\dfrac{a}{b}\cdot \dfrac{c}{d} = \dfrac{ac}{db}$ if $b,d\neq 0$.
         \begin{proof} Using the field axioms
              \begin{align*}
                  \dfrac{a}{b}\cdot \dfrac{c}{d} &= (ab^{-1})\cdot (cd^{-1}) \\
                  &= (ac)\cdot(d^{-1}b^{-1}) \\
                  &= (ac)\cdot (db)^{-1} \\
                  &= \dfrac{ac}{db}
              \end{align*}
         \end{proof}
         \item $\dfrac{a}{b} \bigg/ \dfrac{c}{d} = \dfrac{ad}{bc}$, if $b,d\neq 0$.
         \begin{proof} Using the field axioms
              \begin{align*}
                \dfrac{a}{b} \bigg/ \dfrac{c}{d} &= \dfrac{a}{b}\cdot\left(\dfrac{c}{d}\right)^{-1} \\
                &= ab^{-1}\cdot (cd^{-1})^{-1} \\
                &= ab^{-1} \cdot c^{-1}(d^{-1})^{-1} \\
                &= ab^{-1}\cdot c^{-1}d \\
                &= (ad)\cdot (b^{-1}c^{-1}) \\
                &= (ad)\cdot (bc)^{-1} \\
                &= \frac{ad}{bc}
              \end{align*}
         \end{proof}
         \pagebreak
         \item If $b,d \neq 0$, then $\dfrac{a}{b} = \dfrac{c}{d}$ if and only if
         $ad=bc$. Also determine when $\dfrac{a}{b} = \dfrac{b}{a}$.
         \begin{proof} There are two cases to prove for the first part. 
             \begin{itemize}
                 \item[($\Rightarrow$)] Let $b,d \neq 0$. Assume that
                 $\dfrac{a}{b} = \dfrac{c}{d}$,
                 \begin{align*}
                     \frac{a}{b} &= \frac{c}{d}, \\
                     \qquad ab^{-1} &= cd^{-1 }, \\
                     (ab^{-1})(bd) &= (cd^{-1})(bd), \\
                     (ad)(b\cdot b^{-1}) &= (bc)(d\cot d^{-1}), \\
                     ad &= bc.
                 \end{align*}
                 \item[($\Leftarrow$)] Let $b,d \neq 0$. Assume that $ad=bc$,
                 \begin{align*}
                     ad &= bc, \\
                     (ad)(bd)^{-1} &= (bc)(bd)^{-1} \\
                     (ab^{-1})(d\cdot d^{-1}) &= (cd^{-1})(b\cdot b^{-1}) \\
                     ab^{-1} &= cd^{-1} \\
                     \frac{a}{b} &= \frac{c}{d}
                 \end{align*} 
             \end{itemize}
         \end{proof}
         \begin{proof} From Exercise $1$ Part $(iii)$ we make use of the fact, if
          $x^2=y^2$ then $x=y$ or  $x=-y$.
              \begin{align*}
                  \frac{a}{b} &= \frac{b}{a}, \\
                  ab^{-1} &= ba^{-1}, \\
                  (ab^{-1})(ab) &= (ba^{-1})(ab), \\
                  (a\cdot a)(b\cdot b^{-1}) &= (b\cdot b)(a\cdot a^{-1}), \\
                  a^2 &= b^2.
              \end{align*}
              and so it must be that $a=b$ or $a=-b$.
         \end{proof}
     \end{enumerate}
\end{exercise}
\pagebreak
\begin{exercise}[\textbf{4}] Find all numbers $x$ for which
     \begin{enumerate}
         \item $4-x < 3-2x$. 
         \begin{proof} Using the field axioms
              \begin{align*}
                  4-x &< 3-2x \\
                  4-x+(2x-4) &< 3-2x+(2x-4) \\
                  x &< -1
              \end{align*}
         \end{proof}
         \item $5-x^2 < 8$.
         \begin{proof} Using the field axioms
              \begin{align*}
                  5-x^2+(x^2-5) &< 8+(x^2-5) \\
                  x^2 +3&> 0 
              \end{align*}
              since $x^2\geq 0$ for all $x \in \R$, then it must be that $x^2+3
              > 0$ for all $x \in \R$.
         \end{proof}
         \item $5-x^2<-2$
         \begin{proof} Using the field axioms
              \begin{align*}
                  5-x^2 &< -2 \\
                  x^2 &> 7 \\
                  \abs{x} &> \sqrt{7} \\
                  x < -\sqrt{7} &\text{ or } x>\sqrt{7}
              \end{align*}
         \end{proof}
         \item $(x-3)(x-1)>0$ (When is a product of two numbers positive?)
         \begin{proof} The product of two numbers is postivie if and only if the
         numbers are both positive or both negative. For all $a,b \in \R$, $ab
         >0 \Leftrightarrow a >0 \text{ and } b>0, \text{ or } a<0 \text{ and } b<0$.

         Hence, 
         \begin{equation*}
             x-3>0 \qquad \text{and} \qquad x-1>0
         \end{equation*}
         so it must be that $x>3$. Or 
         \begin{equation*}
             x-3<0 \qquad \text{and} \qquad x-1>0
         \end{equation*}
         and it must be that $x<1$. That is $(x-3)(x-1) > 0$ if $x>3$ or $x<1$.
         \end{proof}
         \pagebreak
         \item $x^2-2x+2 > 0$.
         \begin{proof} Using the field axioms
              \begin{align*}
                  x^2-2x+2 &= (x^2+2x+1)+1 \\
                  &= (x-1)^2+1
              \end{align*}
              for all $x\in \R$ notice that, $(x-1)^2 \geq 0$, so it must be that
              $(x-1)^2+1>0$.
         \end{proof}
         \item $x^2+x+1 > 2$.
         \begin{proof} Using the field axioms
              \begin{align*}
                  x^2+x+1 &> 2 \\
                  x^2+x-1 &> 0 \\
                  (x^2+x+\frac{1}{4})-\frac{5}{4} &> 0 \\
                  \left(x+\frac{1}{2}\right)^2 &> \frac{5}{4} \\
                  \abs{x+\frac{1}{2}} &> \frac{\sqrt{5}}{2} \\
                  x+\frac{1}{2} > \frac{\sqrt{5}}{2} &\text{ or } x+\frac{1}{2}< -\frac{\sqrt{5}}{2}
              \end{align*}
              so it must be that
              \begin{equation*}
                  x > \frac{\sqrt5-1}{2} \qquad \text{or} \qquad x < \frac{-\sqrt{5}-1}{2}
              \end{equation*}
         \end{proof}
         \item $x^2-x+10 > 16$.
         \begin{proof} Using the field axioms
            \begin{align*}
                x^2-x+10 &>16 \\
                x^2-x-6 &> 0 \\
                (x-3)(x+2) &> 0
            \end{align*}
            To assure that the product is positive, it must be that
            the two numbers are both positive or both negative. Hence,
         \begin{equation*}
             x-3>0 \qquad \text{and} \qquad x+2>0
         \end{equation*}
         such that $x>3$. Or 
         \begin{equation*}
             x-3 < 0 \qquad \text{and} \qquad x+2<0
         \end{equation*}
         such that $x<-2$. Therefore, $x^2-x+10>16$ if $x>3$ or $x<-2$.
         \end{proof}
         \pagebreak
         \item $x^2+x+1 > 0$.
         \begin{proof} Using the field axioms
              \begin{align*}
                  x^2+x+1 &= (x^2+x+\frac{1}{4})+\frac{3}{4},\\
                  &= \left(x+\frac{1}{2}\right)^2 + \frac{3}{4} .
              \end{align*}
              for all $x\in \R$, notice that $(x+\frac{1}{2})^2 \geq 0$, so it
              must be that $(x+\frac{1}{2})^2+\frac{3}{4} >0$ for all $x\in \R$.
         \end{proof}
         \item $(x-\pi)(x+5)(x-3)>0$.
         \begin{proof} The expression $(x- \pi)(x+5)(x-3)$ can be rearranged as
         a product of two numbers, namely, $(x- \pi)\left[(x+5)(x-3)\right]$ .
              
         Notice, the product of two real numbers $ab$ is greater than zero if
         $a$ and $b$ are both greater than zero, or both less than zero.

         There are two cases:
         \begin{itemize}
            \item Let $(x- \pi) >0$ so that $x> \pi$, and $(x+5)(x-3)>0$ so that
            $x<-5$ or $x>3$. Therefore it must be that $x> \pi$.
            
            \item Let $(x-\pi) <0$ so that $x< \pi$, and $(x+5)(x-3)<0$ so that
            $-5<x<3$. Therefore it must be that $-5<x<3$. 
         \end{itemize}
         Therefore, $(x-\pi)(x+5)(x-3)>0$ if $x>\pi$ , or $-5<x<3$.   
         \end{proof}
         
         \item $(x-\sqrt[3]{2})(x-\sqrt{2})>0$. 
         \begin{proof} Either both numbers are greater than zero or less than zero.
              \begin{equation*}
                  x>\sqrt[3]{2} \qquad \text{and} \qquad x>\sqrt{2}
              \end{equation*}
              so that $x>\sqrt{2}$. Or 
              \begin{equation*}
                  x<\sqrt[3]{2} \qquad \text{and} \qquad x<\sqrt{2}
              \end{equation*}
              so that $x<\sqrt[3]{2}$.

              Therefore, $(x-\sqrt[3]{2})(x-\sqrt{2})>0$ if $x>\sqrt{2}$ or $x<\sqrt[3]{2}$.
         \end{proof}
         
         \item $2^x<8$.
         \begin{proof} We can rewrite it as 
            \begin{equation*}
                2^x < 2^3 
            \end{equation*}
            Both have the same base, so it must be that the inequality is
            preserved on the exponents.
            \begin{equation*}
                x<3
            \end{equation*}
            so $2^x < 8$, whenever $x<3$.
         \end{proof}
         
         \item $x+3^x<4$.
         \begin{proof} We first notice that $x+3^x=4$ if $x=1$
            \begin{align*}
                x+3^x &= (1)+3^1 \\
                &= 4
            \end{align*}
            observe that $x+3^x$ is always increasing as $x$ increase, and
            decreasing as $x$ decrease. Therefore $x+3^x < 4$ if $x<1$.
         \end{proof}

         \item $\dfrac{1}{x}+\dfrac{1}{1-x}>0$.
         \begin{proof} We can rewrite the expression as
              \begin{align*}
                  \frac{1}{x} + \frac{1}{1-x} &= \frac{(1-x)+x)}{x(1-x)} \\ 
                  &= \frac{1}{x(1-x)}
              \end{align*}
              Notice that $\frac{1}{x(1-x)}>0$, whenever $x(1-x)>0$. So it must
              be that $x$ and $(1-x)$ are greater than zero
              \begin{equation*}
                  x>0 \qquad \text{and} \qquad x<1
              \end{equation*}
              or $x$ and $(1-x)$ are both less than zero
              \begin{equation*}
                  x<0 \qquad \text{and} \qquad x>1
              \end{equation*}
              but there exists no $x$ sucht that $x<0$ and $x>1$. Therefore,
              $\frac{1}{x}+\frac{1}{1-x}>0$ if $x>0$ and $x<1$.
         \end{proof}

         \item $\dfrac{x-1}{x+1}>0$.
         \begin{proof} Either both $(x-1)$ and $(x+1)$ are greater than zero or
         both less than zero.
         \begin{equation*}
             x > 1 \qquad \text{and} \qquad x>-1
         \end{equation*}
         so it must be that $x>1$. Or 
         \begin{equation*}
             x<1 \qquad \text{and} \qquad x<-1
         \end{equation*}
         so it must be that $x<-1$. 
         \end{proof}
     \end{enumerate}
\end{exercise}
\pagebreak
\begin{exercise}[\textbf{5}] Prove the following:
     \begin{enumerate}
         \item If $a<b$ and $c<d$, then $a+c<b+d$.
         \begin{proof} Assume that $a<b$ and $c<d$. Notice that $b-a>0$ and
         $d-c>0$, therefore their sum is also positive, namely,
              \begin{equation*}
                  (b-a)+(d-c)>0
              \end{equation*}
            so that 
            \begin{equation*}
                a+c < b+d
            \end{equation*}
         \end{proof}
         \item If $a<b$, then $-b<-a$.
         \begin{proof} Assume that $a<b$. Therefore, $a-b<0$. Notice that,
            \begin{align*}
                -(a-b)&<0, \\
                b-a&<0, \\
                -a &< -b.                
            \end{align*}              
         \end{proof}
         \item If $a<b$ and $c>d$, then $a-c<b-d$.
         \begin{proof}Assume that $a<b$ and $c>d$. Therefore, $a-b<0$ and $c-d>$
         so that $a-b<0<c-d$. Therefore,
              \begin{align*}
                a-b&<c-d, \\
                a-c &<b-d.
              \end{align*}
         \end{proof}
         \item If $a<b$ and $c>0$, then $ac<bc$.
         \begin{proof}
              
         \end{proof}
         \item If $a<b$ and $c<0$, then $ac>bc$.
         \begin{proof}
              
         \end{proof}
         \item If $a>1$, then $a^2>a$.
         \begin{proof}
              
         \end{proof}
         \item If $0<a<1$, then $a^2<a$.
         \begin{proof}
              
         \end{proof}
         \item If $0\leq a<b$ and $0\leq c<d$, then $ac<bd$.
         \begin{proof}
              
         \end{proof}
         \item If $0 \leq a<b$, then $a^2<b^2$.
         \begin{proof}
              
         \end{proof}
         \item If $a,b \geq 0$ and $a^2<b^2$, then $a<b$.
     \end{enumerate}
\end{exercise}