\section{Prologue}
\subsection{Basic Properties of Numbers}
\begin{defi}[Field Properties] The following properties hold in $\R$
    
     \begin{itemize}
         \item[P1](Associative law for addition) \hfill $a+(b+c) = (a+b)+c$.
         \item[P2](Existence of an additive identity) \hfill $a+0=0+a=a$.
         \item[P3](Existence of additive inverse) \hfill $a+(-a)=(-a)+a=0 $.
         \item[P4](Commutative law for addition) \hfill $a+b=b+a$.
         \item[P5](Associative law for multiplication) \hfill $a\cdot(b\cdot c)
         = (a\cdot b)\cdot c$.
         \item[P6](Existence of multiplicative identity) \hfill $a\cdot 1 =
         1\cdot a = a;\quad 1 \neq 0$.
         \item[P7](Existence of multiplicative inverses) \hfill $a\cdot a^{-1} =
         a^{-1}\cdot a = 1,\text{ for } a \neq 0$.
         \item[P8](Commutative law for multiplication) \hfill $a\cdot b = b\cdot a$.
         \item[P9](Distributive law) \hfill $a\cdot(b+c) = a\cdot b+ a\cdot c$.
         \item[P10](Trichotomy law) For every number $a$, one and only one of
         the following holds: (Denote $P$ as the collection of positive numbers)
            \begin{enumerate}
                \item $a=0$,
                \item $a $ is in the collection $P$,
                \item $-a $ is in the collection $P$.
            \end{enumerate}
        \item[P11](Closure under addition) If $a $ and $ b $ are in $ P $, then
        $a+b$ is in $P$.
        \item[P12](Closure under multiplication) If $a$ and $b$ are in $P$, then
        $a\cdot b$ is in $P$.
     \end{itemize}
\end{defi}
\begin{defi}[Absolute Value] For any number $a$, we define the \emph{absolute
value} $\abs{a}$ of $a$ as follows:
\[
    \abs{a} = 
    \begin{cases}
         a, \hfill &\quad  a \geq 0 \\
         -a, \hfill &\quad a \leq 0 
    \end{cases}
\]
\end{defi}
\begin{thm}[Triangle Inequality]
     For all numbers $a$ and $b$, we have
     \begin{equation*}
         \abs{a+b} \leq \abs{a}+\abs{b}
     \end{equation*}
\end{thm}
\begin{proof} We make use of the fact that if both $x$ and $y$ are nonnegative,
then $x^2 < y^2$ implies $x < y$.
      \begin{align*}
          \abs{a+b}^2 &= a^2+2ab+b^2 \\
          &= \abs{a}^2+2ab+\abs{b}^2 \\
          &\leq \abs{a}^2+2\abs{a}\abs{b}+\abs{b}^2 \\
          &=(\abs{a}+\abs{b})^2 
      \end{align*}
      Since $\abs{a+b}$ and $(\abs{a}+\abs{b})$ are both nonnegative, then
      \begin{equation*}
          \abs{a+b} \leq \abs{a}+\abs{b}.
      \end{equation*}
\end{proof}

\subsubsection{Exercises}
\begin{exercise}[\textbf{1}]
     Prove the following:
    \begin{enumerate}
        \item If $ax=a$ for some number $a \neq 0$, then $x=1$.
        \begin{proof}
            Assume that $ax=a$ fro some number $a\neq 0$. 
            \begin{align*}
                x = x\cdot 1 = x\cdot (a\cdot a^{-1}) &= ax \cdot (a^{-1}) \\
                &= a \cdot (a^{-1}) \\
                &= (a\cdot a^{-1}) \\   
                &= 1
            \end{align*}
        \end{proof}
        \item $x^2-y^2 = (x-y)(x+y)$.
        \begin{proof} Using the field axioms.
             \begin{align*}
                 (x-y)(x+y) &= x\cdot (x+y) + (-y)\cdot (x+y) \\
                 &= (x^2+xy)+((-y)\cdot x+(-y)\cdot y) \\
                 &= x^2+xy-xy-y^2 \\
                 &= x^2-y^2
             \end{align*}
        \end{proof}
        \item If $x^2=y^2$, then $x=y$ or $x=-y$.
        \begin{proof}Assume that $x^2=y^2$. We make use of (ii).
             \begin{align*}
                 x^2 = y^2 &\Leftrightarrow  x^2-y^2 = 0 \\
                 & \Leftrightarrow (x-y)(x+y) =0 \\
                 & \Rightarrow x=y \text{ or } x=-y.
             \end{align*}
        \end{proof}
        \item $x^3-y^3=(x-y)(x^2+xy+y^2)$.
        \begin{proof}
             Using the field axioms
             \begin{align*}
                 (x-y)(x^2+xy+y^2) &= x^2(x-y)+xy(x-y)+y^2(x-y) \\
                 &= (x^3-x^2y)+(x^2y-xy^2)+(xy^2-y^3) \\
                 &= x^3+(x^2y-x^2y)+(xy^2-xy^2)-y^3 \\
                 &= x^3-y^3
             \end{align*}
        \end{proof}
        \item $x^n-y^n = (x-y)(x^{n-1}+x^{n-2}y+\dotsb +xy^{n-2}+y^{n-1}).$
        \begin{proof}Using the field axioms
             \begin{align*}
                 (x-y)(x^{n-1}+x^{n-2}y+\dotsb +xy^{n-2}+y^{n-1}) &= x(x^{n-1}+x^{n-2}y+\dotsb+xy^{n-2}+y^{n-1}) \\
                 & \qquad -[y(x^{n-1}+x^{n-2}y+\dotsb+xy^{n-2}+y^{n-1})] \\
                 &= x^n+x^{n-1}y+\dotsb+x^2y^{n-2}+xy^{n-1} \\
                 & \qquad -[x^{n-1}y+x^{n-2}y^2+\dotsb+xy^{n-1}+y^{n}] \\
                 &= x^n-y^n
             \end{align*}
        \end{proof}
        \begin{proof}[Alternative Proof] We make us of sigma notation
             \begin{align*}
                 (x-y)\cdot \sum_{i=0}^{n-1}x^iy^{n-(i+1)} &= x\left(\sum_{i=0}^{n-1}x^iy^{n-(i+1)}\right)- \left[y\left(\sum_{i=0}^{n-1}x^iy^{n-(i+1)}\right)\right] \\
                 &= \sum_{i=0}^{n-1}x^{i+1}y^{n-(i+1)} -\left[\sum_{i=0}^{n-1}x^iy^{n-i}\right] \\
                 &= x^n +\sum_{i=0}^{n-2}x^{i+1}y^{n-(i+1)} -\left[\sum_{i=1}^{n-1}x^iy^{n-i} +y^n\right] \\
                 &= x^n +\sum_{i=0}^{n-2}x^{i+1}y^{n-(i+1)} -\left[\sum_{i=0}^{n-2}x^{i+1}y^{n-(i+1)} +y^n\right] \\
                 &= x^n-y^n + \sum_{i=0}^{n-2} \left[x^{i+1}y^{n-(i+1)} - (x^{i+1}y^{n-(i+1)}) \right] \\
                 &= x^n-y^n +\sum_{i=0}^{n-2} 0 \\
                 &= x^n-y^n
             \end{align*}
        \end{proof}
        \item $x^3+y^3 = (x+y)(x^2-xy+y^2).$
        \begin{proof} Replace $y$ by $-y$ in part (iv)
             \begin{align*}
                x^3-y^3 = (x-y)(x^2+xy+y^2) & \Leftrightarrow x^3-(-y)^3 = (x-(-y))(x^2+x(-y)+(-y)^2) \\
                & \Leftrightarrow x^3+y^3 = (x+y)(x^2-xy+y^2)
             \end{align*}
        \end{proof}
    \end{enumerate}
\end{exercise}

\begin{exercise}[\textbf{2}]
     What is wrong with the following "proof"? Let $x=y$. Then
     \begin{align*}
         x^2 &= xy, \\
         x^2-y^2 &= xy-y^2, \\
         (x+y)(x-y) &= y(x-y), \\
         x+y &= y, \\
         2y &= y, \\ 
         2 &= 1.    
     \end{align*}
     \begin{proof}[Solution] For all $a \in \R$ we know that $a\cdot a^{-1}=0$
     with the assumption $a\neq 0$. The $4th$ step is contradictory on the given
     fact that $x=y$ which implies $x-y=0$ and has no multiplicative inverse.
    \end{proof}
\end{exercise}
\begin{exercise}[\textbf{3}] Prove the following:
     \begin{enumerate}
         \item $\dfrac{a}{b} = \dfrac{ac}{bc}$, if $b,c \neq 0$.
         \begin{proof} Using the field axioms
              \begin{align*}
                  \dfrac{a}{b} = ab^{-1} &= (ab^{-1})(c\cdot c^{-1}) \\
                  &= (ac)(b^{-1}c^{-1}) \\
                  &= (ac)(bc)^{-1} \\
                  &= \dfrac{ac}{bc}
              \end{align*}
         \end{proof}
         \item $\dfrac{a}{b}+\dfrac{c}{d} = \dfrac{ad+bc}{bd}$, if $b,d \neq 0$.
         \begin{proof} Using the field axioms   
              \begin{align*}
                  \dfrac{a}{b}+\dfrac{c}{d} = ab^{-1}+cd^{-1} &= (ab^{-1}+cd^{-1})\cdot (bd)(bd)^{-1} \\
                  &= (ad(b\cdot b^{-1})+bc(d\cdot d^{-1}))\cdot (bd)^{-1} \\
                  &= (ad+bc)\cdot (bd)^{-1} \\
                  &= \dfrac{ad+bc}{bd}
              \end{align*}
         \end{proof}
         \pagebreak
         \item $(ab)^{-1} = a^{-1}b^{-1}$, if $a,b\neq 0$. 
         \begin{proof} Using the field axioms
              \begin{align*}
                  ab(a^{-1}b^{-1}) &=  1 \\
                  a^{-1}b^{-1} &= (ab)^{-1}
              \end{align*}
         \end{proof}
         \item $\dfrac{a}{b}\cdot \dfrac{c}{d} = \dfrac{ac}{db}$ if $b,d\neq 0$.
         \begin{proof} Using the field axioms
              \begin{align*}
                  \dfrac{a}{b}\cdot \dfrac{c}{d} &= (ab^{-1})\cdot (cd^{-1}) \\
                  &= (ac)\cdot(d^{-1}b^{-1}) \\
                  &= (ac)\cdot (db)^{-1} \\
                  &= \dfrac{ac}{db}
              \end{align*}
         \end{proof}
         \item $\dfrac{a}{b} \bigg/ \dfrac{c}{d} = \dfrac{ad}{bc}$, if $b,d\neq 0$.
         \begin{proof} Using the field axioms
              \begin{align*}
                \dfrac{a}{b} \bigg/ \dfrac{c}{d} &= \dfrac{a}{b}\cdot\left(\dfrac{c}{d}\right)^{-1} \\
                &= ab^{-1}\cdot (cd^{-1})^{-1} \\
                &= ab^{-1} \cdot c^{-1}(d^{-1})^{-1} \\
                &= ab^{-1}\cdot c^{-1}d \\
                &= (ad)\cdot (b^{-1}c^{-1}) \\
                &= (ad)\cdot (bc)^{-1} \\
                &= \frac{ad}{bc}
              \end{align*}
         \end{proof}
         \pagebreak
         \item If $b,d \neq 0$, then $\dfrac{a}{b} = \dfrac{c}{d}$ if and only if
         $ad=bc$. Also determine when $\dfrac{a}{b} = \dfrac{b}{a}$.
         \begin{proof} There are two cases to prove for the first part. 
             \begin{itemize}
                 \item[($\Rightarrow$)] Let $b,d \neq 0$. Assume that
                 $\dfrac{a}{b} = \dfrac{c}{d}$,
                 \begin{align*}
                     \frac{a}{b} &= \frac{c}{d}, \\
                     \qquad ab^{-1} &= cd^{-1 }, \\
                     (ab^{-1})(bd) &= (cd^{-1})(bd), \\
                     (ad)(b\cdot b^{-1}) &= (bc)(d\cot d^{-1}), \\
                     ad &= bc.
                 \end{align*}
                 \item[($\Leftarrow$)] Let $b,d \neq 0$. Assume that $ad=bc$,
                 \begin{align*}
                     ad &= bc, \\
                     (ad)(bd)^{-1} &= (bc)(bd)^{-1} \\
                     (ab^{-1})(d\cdot d^{-1}) &= (cd^{-1})(b\cdot b^{-1}) \\
                     ab^{-1} &= cd^{-1} \\
                     \frac{a}{b} &= \frac{c}{d}
                 \end{align*} 
             \end{itemize}
         \end{proof}
         \begin{proof} From Exercise $1$ Part $(iii)$ we make use of the fact, if
          $x^2=y^2$ then $x=y$ or  $x=-y$.
              \begin{align*}
                  \frac{a}{b} &= \frac{b}{a}, \\
                  ab^{-1} &= ba^{-1}, \\
                  (ab^{-1})(ab) &= (ba^{-1})(ab), \\
                  (a\cdot a)(b\cdot b^{-1}) &= (b\cdot b)(a\cdot a^{-1}), \\
                  a^2 &= b^2.
              \end{align*}
              and so it must be that $a=b$ or $a=-b$.
         \end{proof}
     \end{enumerate}
\end{exercise}
\pagebreak
\begin{exercise}[\textbf{4}] Find all numbers $x$ for which
     \begin{enumerate}
         \item $4-x < 3-2x$. 
         \begin{proof} Using the field axioms
              \begin{align*}
                  4-x &< 3-2x \\
                  4-x+(2x-4) &< 3-2x+(2x-4) \\
                  x &< -1
              \end{align*}
         \end{proof}
         \item $5-x^2 < 8$.
         \begin{proof} Using the field axioms
              \begin{align*}
                  5-x^2+(x^2-5) &< 8+(x^2-5) \\
                  x^2 +3&> 0 
              \end{align*}
              since $x^2\geq 0$ for all $x \in \R$, then it must be that $x^2+3
              > 0$ for all $x \in \R$.
         \end{proof}
         \item $5-x^2<-2$
         \begin{proof} Using the field axioms
              \begin{align*}
                  5-x^2 &< -2 \\
                  x^2 &> 7 \\
                  \abs{x} &> \sqrt{7} \\
                  x < -\sqrt{7} &\text{ or } x>\sqrt{7}
              \end{align*}
         \end{proof}
         \item $(x-3)(x-1)>0$ (When is a product of two numbers positive?)
         \begin{proof} The product of two numbers is postivie if and only if the
         numbers are both positive or both negative. For all $a,b \in \R$, $ab
         >0 \Leftrightarrow a >0 \text{ and } b>0, \text{ or } a<0 \text{ and } b<0$.

         Hence, 
         \begin{equation*}
             x-3>0 \qquad \text{and} \qquad x-1>0
         \end{equation*}
         so it must be that $x>3$. Or 
         \begin{equation*}
             x-3<0 \qquad \text{and} \qquad x-1>0
         \end{equation*}
         and it must be that $x<1$. That is $(x-3)(x-1) > 0$ if $x>3$ or $x<1$.
         \end{proof}
         \pagebreak
         \item $x^2-2x+2 > 0$.
         \begin{proof} Using the field axioms
              \begin{align*}
                  x^2-2x+2 &= (x^2+2x+1)+1 \\
                  &= (x-1)^2+1
              \end{align*}
              for all $x\in \R$ notice that, $(x-1)^2 \geq 0$, so it must be that
              $(x-1)^2+1>0$.
         \end{proof}
         \item $x^2+x+1 > 2$.
         \begin{proof} Using the field axioms
              \begin{align*}
                  x^2+x+1 &> 2 \\
                  x^2+x-1 &> 0 \\
                  (x^2+x+\frac{1}{4})-\frac{5}{4} &> 0 \\
                  \left(x+\frac{1}{2}\right)^2 &> \frac{5}{4} \\
                  \abs{x+\frac{1}{2}} &> \frac{\sqrt{5}}{2} \\
                  x+\frac{1}{2} > \frac{\sqrt{5}}{2} &\text{ or } x+\frac{1}{2}< -\frac{\sqrt{5}}{2}
              \end{align*}
              so it must be that
              \begin{equation*}
                  x > \frac{\sqrt5-1}{2} \qquad \text{or} \qquad x < \frac{-\sqrt{5}-1}{2}
              \end{equation*}
         \end{proof}
         \item $x^2-x+10 > 16$.
         \begin{proof} Using the field axioms
            \begin{align*}
                x^2-x+10 &>16 \\
                x^2-x-6 &> 0 \\
                (x-3)(x+2) &> 0
            \end{align*}
            To assure that the product is positive, it must be that
            the two numbers are both positive or both negative. Hence,
         \begin{equation*}
             x-3>0 \qquad \text{and} \qquad x+2>0
         \end{equation*}
         such that $x>3$. Or 
         \begin{equation*}
             x-3 < 0 \qquad \text{and} \qquad x+2<0
         \end{equation*}
         such that $x<-2$. Therefore, $x^2-x+10>16$ if $x>3$ or $x<-2$.
         \end{proof}
         \pagebreak
         \item $x^2+x+1 > 0$.
         \begin{proof} Using the field axioms
              \begin{align*}
                  x^2+x+1 &= (x^2+x+\frac{1}{4})+\frac{3}{4},\\
                  &= \left(x+\frac{1}{2}\right)^2 + \frac{3}{4} .
              \end{align*}
              for all $x\in \R$, notice that $(x+\frac{1}{2})^2 \geq 0$, so it
              must be that $(x+\frac{1}{2})^2+\frac{3}{4} >0$ for all $x\in \R$.
         \end{proof}
         \item $(x-\pi)(x+5)(x-3)>0$.
         \begin{proof} The expression $(x- \pi)(x+5)(x-3)$ can be rearranged as
         a product of two numbers, namely, $(x- \pi)\left[(x+5)(x-3)\right]$ .
              
         Notice, the product of two real numbers $ab$ is greater than zero if
         $a$ and $b$ are both greater than zero, or both less than zero.

         There are two cases:
         \begin{itemize}
            \item Let $(x- \pi) >0$ so that $x> \pi$, and $(x+5)(x-3)>0$ so that
            $x<-5$ or $x>3$. Therefore it must be that $x> \pi$.
            
            \item Let $(x-\pi) <0$ so that $x< \pi$, and $(x+5)(x-3)<0$ so that
            $-5<x<3$. Therefore it must be that $-5<x<3$. 
         \end{itemize}
         Therefore, $(x-\pi)(x+5)(x-3)>0$ if $x>\pi$ , or $-5<x<3$.   
         \end{proof}
         
         \item $(x-\sqrt[3]{2})(x-\sqrt{2})>0$. 
         \begin{proof} Either both numbers are greater than zero or less than zero.
              \begin{equation*}
                  x>\sqrt[3]{2} \qquad \text{and} \qquad x>\sqrt{2}
              \end{equation*}
              so that $x>\sqrt{2}$. Or 
              \begin{equation*}
                  x<\sqrt[3]{2} \qquad \text{and} \qquad x<\sqrt{2}
              \end{equation*}
              so that $x<\sqrt[3]{2}$.

              Therefore, $(x-\sqrt[3]{2})(x-\sqrt{2})>0$ if $x>\sqrt{2}$ or $x<\sqrt[3]{2}$.
         \end{proof}
         
         \item $2^x<8$.
         \begin{proof} We can rewrite it as 
            \begin{equation*}
                2^x < 2^3 
            \end{equation*}
            Both have the same base, so it must be that the inequality is
            preserved on the exponents.
            \begin{equation*}
                x<3
            \end{equation*}
            so $2^x < 8$, whenever $x<3$.
         \end{proof}
         
         \item $x+3^x<4$.
         \begin{proof} We first notice that $x+3^x=4$ if $x=1$
            \begin{align*}
                x+3^x &= (1)+3^1 \\
                &= 4
            \end{align*}
            observe that $x+3^x$ is always increasing as $x$ increase, and
            decreasing as $x$ decrease. Therefore $x+3^x < 4$ if $x<1$.
         \end{proof}

         \item $\dfrac{1}{x}+\dfrac{1}{1-x}>0$.
         \begin{proof} We can rewrite the expression as
              \begin{align*}
                  \frac{1}{x} + \frac{1}{1-x} &= \frac{(1-x)+x)}{x(1-x)} \\ 
                  &= \frac{1}{x(1-x)}
              \end{align*}
              Notice that $\frac{1}{x(1-x)}>0$, whenever $x(1-x)>0$. So it must
              be that $x$ and $(1-x)$ are greater than zero
              \begin{equation*}
                  x>0 \qquad \text{and} \qquad x<1
              \end{equation*}
              or $x$ and $(1-x)$ are both less than zero
              \begin{equation*}
                  x<0 \qquad \text{and} \qquad x>1
              \end{equation*}
              but there exists no $x$ sucht that $x<0$ and $x>1$. Therefore,
              $\frac{1}{x}+\frac{1}{1-x}>0$ if $x>0$ and $x<1$.
         \end{proof}

         \item $\dfrac{x-1}{x+1}>0$.
         \begin{proof} Either both $(x-1)$ and $(x+1)$ are greater than zero or
         both less than zero.
         \begin{equation*}
             x > 1 \qquad \text{and} \qquad x>-1
         \end{equation*}
         so it must be that $x>1$. Or 
         \begin{equation*}
             x<1 \qquad \text{and} \qquad x<-1
         \end{equation*}
         so it must be that $x<-1$. 
         \end{proof}
     \end{enumerate}
\end{exercise}
\pagebreak
\begin{exercise}[\textbf{5}] Prove the following:
     \begin{enumerate}
         \item If $a<b$ and $c<d$, then $a+c<b+d$.
         \begin{proof} Assume that $a<b$ and $c<d$. Notice that $b-a>0$ and
         $d-c>0$, therefore their sum is also positive, namely,
              \begin{equation*}
                  (b-a)+(d-c)>0
              \end{equation*}
            so that 
            \begin{equation*}
                a+c < b+d
            \end{equation*}
         \end{proof}
         \item If $a<b$, then $-b<-a$.
         \begin{proof} Assume that $a<b$. Therefore, $a-b<0$. Notice that,
            \begin{align*}
                -(a-b)&<0, \\
                b-a&<0, \\
                -a &< -b.                
            \end{align*}              
         \end{proof}
         \item If $a<b$ and $c>d$, then $a-c<b-d$.
         \begin{proof}Assume that $a<b$ and $c>d$. Therefore, $a-b<0$ and $c-d>$
         so that $a-b<0<c-d$. Therefore,
              \begin{align*}
                a-b&<c-d, \\
                a-c &<b-d.
              \end{align*}
         \end{proof}
         \item If $a<b$ and $c>0$, then $ac<bc$.
         \begin{proof} Assume that $a<b$ and $c>0$. Therefore, $b-a>0$ and their
         product is positive,
              \begin{align*}
                  (b-a)\cdot c&>0, \\
                  bc-ac &> 0, \\
                  ac &< bc.
              \end{align*}
         \end{proof}
         \item If $a<b$ and $c<0$, then $ac>bc$.
         \begin{proof} Assume that $a<b$ and $c<0$. Then $b-a>0$ and $0-c =
         -c>0$, and their product must be positive,
              \begin{align*}
                  (b-a)\cdot -c &> 0, \\
                  -bc+ac &> 0, \\
                  ac&> bc.
              \end{align*}
         \end{proof}
         \item If $a>1$, then $a^2>a$.
         \begin{proof} Assume that $a>1$. Notice that $a-1>0$ and $a>0$, so
         their product is positive,
              \begin{align*}
                  (a-1)\cdot a &> 0, \\
                  a^2-a &>0,\\
                  a^2 &> a.
              \end{align*}
         \end{proof}
         \item If $0<a<1$, then $a^2<a$.
         \begin{proof} Assume that $0<a<1$, therefore $a>0$ and $1-a >0$ so that
         their product is also positive,
              \begin{align*}
                  (1-a)\cdot a &>0, \\
                    a-a^2 &> 0, \\
                    a^2 &< a.
              \end{align*}
         \end{proof}
         \item If $0\leq a<b$ and $0\leq c<d$, then $ac<bd$.
         \begin{proof}Assume that $0\leq a<b$ and $0\leq c<d$. Notice that
         $bd>0$, and if either one of $a$ or $c$ is equal to zero so that $ac$ is zero, then
              \begin{equation*}
                  0=ac<bd.
              \end{equation*}
              Otherwise, if $a$ and $c$ are greater than zero then $ac$ is also greater than zero,
              \begin{equation*}
                  0<ac<bc<bd.
              \end{equation*}
         \end{proof}
         \item If $0 \leq a<b$, then $a^2<b^2$.
         \begin{proof}Assume that $0 \leq a<b$. If $a=0$, then $a^2 = 0$ so that 
              \begin{equation*}
                  a^2 < b^2.
              \end{equation*}
              Suppose that $a>0$. From our assumption, $a<b$ therefore
              \begin{equation*}
                a^2 < ab < b^2.
              \end{equation*}
         \end{proof}
         \item If $a,b \geq 0$ and $a^2<b^2$, then $a<b$.
         \begin{proof}Suppose for contradiction that $a,b \geq 0$ and $a^2<b^2$,
         but $a \geq b$. So either $a=b$ or $a>b$.

         If $a=b$, then $a^2=b^2$, a contradiction. Now if $a>b \geq 0$, then 
            \begin{equation*}
                a^2 > ab> b^2
            \end{equation*}
            also a contradiction. Therefore it must be that $a<b$.
         \end{proof}
     \end{enumerate}
\end{exercise}
\pagebreak
\begin{exercise}[\textbf{6}] Prove the following:
     \begin{enumerate}
         \item Prove that if $0 \leq x<y$, then $x^n < y^n$, $n=1,2,3,\dotsc$
         \begin{proof} Assume that $0 \leq x < y$. From the previous problems,
         it must be that $x^2<y^2$, and so on for all $n \in \N$.
              
         \end{proof}
         \item Prove that if $x<y$ and $n$ is odd, then $x^n<y^n$.
         \begin{proof}Assume tha $x<y$ and $n$ is odd. There are three cases:
            \begin{itemize}
                \item If $0 \leq x$, then by the previous exercise, it must be that
                \begin{equation*}
                    x^n < y^n.
                \end{equation*} 
                \item If $x<y\leq 0$, then it must be that $0 \leq -y < -x$. Therefore, 
                \begin{align*}
                    (-y)^n &< (-x)^n, \\
                    -y^n &< -x^n, \\
                    x^n &< y^n.
                \end{align*}
                \item If $x<0 \leq y$, then since $n$ is odd, it must be that 
                \begin{equation*}
                    x^n < 0 \leq y^n
                \end{equation*}
                so that $x^n < y^n$.
            \end{itemize}
              
         \end{proof}
         \item Prove that if $x^n = y^n$ and $n$ is odd, then $x=y$.(
         \begin{proof} 
            If it is not, then it must be that $x^n < y^n$ or $x^n > y^n$.
              
         \end{proof}
         \item Prove that if $x^n=y^n$ and $n$ is even, then $x=y$ or $x=-y$.
         \begin{proof} We have three cases and we make use of (i): 
            \begin{itemize}
                \item Let $x,y \geq 0$. If $x^n=y^n$, then $x=y$. Since from
                part (i), without loss of generality, $x\neq y$ implies that
                $x^n < y^n$ for all $n \in \N$.
                \item Let $x,y \leq 0$ so that $-x, -y \geq 0$. If $(-x)^n =
                (-y)^n$, then $-x=-y$  so that $x=y$ for all $n\in \N$.
                \item Let $x\leq 0, y \geq 0 $ so that $-x, y \geq 0$. If
                $(-x)^n = y^n$, then $-x = y$ which is the same as $x=-y$, for all $n \in \N$.
            \end{itemize}
            We have now exhausted all the cases of $x$ and $y$. Therefore, if
            $x^n=y^n$ where $n$ is even, then $x=y$ or $x=-y$. 
         \end{proof}
\end{enumerate}
\end{exercise}
\pagebreak
\begin{exercise}[\textbf{7}] Prove that if $0<a<b$, then 
\begin{equation*}
    a<\sqrt{ab}<\frac{a+b}{2}<b.
\end{equation*}
Notice that the inequality $\sqrt{ab} \leq (a+b)/2$ holds for all $a,b \geq 0$.
A generalization of this fact occurs in Exercise $2-22$.
\begin{proof} Since $0<a<b$, then we know that 
     \begin{equation*}
         a^2 = a\cdot a < ab < b\cdot b = b^2.
     \end{equation*}
     therefore $a < \sqrt{ab}<b$.

     Notice also that 
     \begin{equation*}
         a = \frac{a+a}{2} < \frac{a+b}{2} < \frac{b+b}{2} = b.
     \end{equation*}
     so that $a < (a+b)/2$ and $b > (a+b)/2$.

     Lastly, we must show that $\sqrt{ab} < (a+b)/2$. But since $a,b >0$, we
     know that 
     \begin{align*}
         (a-b)^2 &>0, \\
         a^2-2ab+b^2 &> 0, \\
         a^2+2ab+b^2 &> 4ab, \\
         (a+b)^2 &> 4ab, \\
         \frac{a+b}{2} &> \sqrt{ab}
     \end{align*}
     for all $a,b$ such that $0<a<b$.
\end{proof}
\end{exercise}
\begin{exercise}[\textbf{12}] Prove the following:
     \begin{enumerate}
         \item $\abs{xy} = \abs{x} \cdot \abs{y}$.
         \begin{proof} We will prove this fact the same way we proved the
         \emph{Triangle Inequality}, by using that fact that if $x^2 = y^2$ and
         $x,y$ are nonnegative, then $x=y$. Notice
              \begin{equation*}
                  |xy|^2 = (xy)^2 = x^2y^2 = |x|^2\cdot |y|^2 = (|x|\cdot |y|)^2.
              \end{equation*}
        Since $|xy|$ and $|x|\cdot |y|$ are always nonnegative, we find that 
        \begin{equation*}
            |xy| = |x|\cdot |y|.
        \end{equation*}
         \end{proof}
         \item $\left| \dfrac{1}{x} \right|  = \dfrac{1}{\abs{x}}$, if $x\neq
         0$.
         \begin{proof} We make use of the previous exercise.
              \begin{equation*}
                \left| \frac{1}{x}\right| \cdot |x| = \left| \frac{1}{x}\cdot x\right| =  |1| = 1
              \end{equation*}
              therefore
              \begin{equation*}
                  \left| \frac{1}{x} \right| = \frac{1}{|x|}.
              \end{equation*}
         \end{proof}
         \item $\dfrac{|x|}{|y|} = \left| \dfrac{x}{y}\right|$, if $y\neq 0$.
         \begin{proof} It follows by using the result of the previous exercise,
            \begin{equation*}
                |x^{-1}| = |x|^{-1}.
            \end{equation*}
         \end{proof}
         \item $|x-y| \leq |x|+|y|$.
         \begin{proof} We make use of the \emph{Triangle Inequality} and the
         fact that for all $x\in \R$, $|x|=|-x|$. Let $y_0 = -y$, 
              \begin{align*}
                  |x+y_0| &\leq |x|+|y_0|, \\
                  |x+(-y)| &\leq |x|+|-y|, \\
                  |x-y| &\leq |x|+|y|.
              \end{align*}
         \end{proof}
         \item $|x|-|y| \leq |x-y|$.
         \begin{proof} From the \emph{Triangle Inequality},
              \begin{align*}
                  |x| &= |(x-y)+y| \\
                  & \leq |x-y|+|y|
              \end{align*}
              so that
              \begin{equation*}
                  |x|-|y| \leq |x-y|.
              \end{equation*}
         \end{proof}
         \item $|(|x|-|y|)| \leq |x-y|$.
         \begin{proof} To show that $|(|x|-|y|)| \leq |x-y|$, we must show that
         $-|x-y| \leq |x|-|y| \leq |x-y|$.
              \begin{itemize}
                  \item The second inequality follows from the previous exercise. 
                  \item The first inequality is the same as $|y|-|x| \leq |x-y|$, but this
                  also follows from the previous exercises since $|y|-|x| \leq |y-x|=|x-y|$.
              \end{itemize}
         \end{proof}
         \item $|x+y+z| \leq |x|+|y|+|z|$.
         \begin{proof} We just apply the \emph{Triangle Inequality} multiple times.
              \begin{align*}
                |x+y+z| &\leq |x|+|y+z| \\
                &\leq |x|+|y|+|z|.
              \end{align*}
         \end{proof}
     \end{enumerate}
\end{exercise}
\pagebreak
\begin{exercise}[\textbf{14}] Prove the following
     \begin{enumerate}
        \item Prove that $|a|=|-a|$.
        \begin{proof} If $a \geq 0$, then $-a \leq 0$ and its absolute value is
             \begin{equation*}
                 |-a| = -(-a) = a = |a|
             \end{equation*}
             Now, if $a \leq 0$, then $-a \geq 0$ so it follows from the
             previous one that $|a|=|-a|$.
        \end{proof}
        \item Prove that $-b \leq a \leq b$ if and only if $|a| \leq b$. In
        particular, it follows that $-|a| \leq a \leq |a|$.
        \begin{proof} This is a biconditional, so we prove it in both directions.
             
            Assume that $-b \leq a \leq b$. If $a \geq 0$, then 
            \begin{equation*}
                |a| = a \leq b.
            \end{equation*}
            If $a \leq 0$ and also notice that $-a \leq b$, then
            \begin{equation*}
                |a|=-a \leq b.
            \end{equation*}

            For the converse, we assume that $|a|\leq b$ and it follows that $b\geq0$. If $a\geq 0$, then
            \begin{equation*}
                a=|a|\leq b.
            \end{equation*}   
            and 
            \begin{equation*}
                -b \leq 0 \leq a.
            \end{equation*}
            Now, if $a\leq 0$, then $-a = |a|\leq b$ so that 
            \begin{equation*}
                -b \leq a.
            \end{equation*}
            and
            \begin{equation*}
                a \leq 0 \leq b.
            \end{equation*}
        \end{proof}
        \item Use this fact to give a new proof that $|a+b| \leq |a|+|b|$.
        \begin{proof} We make use of the fact that $-|a| \leq a \leq |a|$ and
        $-|b| \leq b \leq |b|$, and it follows that
             \begin{equation*}
                 -(|a|+|b|) \leq a+b \leq |a|+|b|,
             \end{equation*}
             so $|a+b| \leq |a| + |b|$.
        \end{proof}
     \end{enumerate}
\end{exercise}
\pagebreak
\begin{exercise}[\textbf{18}] Prove the following
     \begin{enumerate}
         \item Suppose that $b^2-4c \geq 0$. Show that the numbers 
         \begin{equation*}
             \frac{-b+\sqrt{b^2-4c}}{2}, \qquad \frac{-b-\sqrt{b^2-4c}}{2}
         \end{equation*}
         both satisfy the equation $x^2+bx+c=0$.
         \begin{proof} From the equation, we can find that
              \begin{align*}
                  x^2+bx+c &= 0, \\
                  \left(x+ \frac{b}{2}\right)^2 &= -c+\frac{b^2}{4}, \\
                  \left| x+\frac{b}{2} \right| &= \frac{\sqrt{b^2-4c}}{|2|}, \\
                  x &= \frac{-b\pm \sqrt{b^2-4c}}{2}.
              \end{align*}
         \end{proof}
         \item Suppose that $b^2-4c < 0$. Show there are no numbers $x$
         satisfying $x^2+bx+c=0$; in fact, $x^2+bx+c > 0$ for all $x$.
         \begin{proof} We complete the square
              \begin{align*}
                  x^2+bx+c &= \left( x+\frac{b}{2}\right)^2 + c-\frac{b^2}{4} \\
                  &= \left( x+\frac{b}{2}\right)^2-\left( \frac{b^2-4c}{4}\right)
              \end{align*}
              We know that squares are always nonnegative, and also we know that
              $b^2-4c <0$ so the second term is always positive. Therefore
              $x^2+bx+c$ is always positive for all $x$.
         \end{proof}
     \item Use this fact to give another proof that if $x$ and $y$ are not both
     $0$, then $x^2+xy+y^2>0$.
     \begin{proof} From the previous exercise, we make $y^2-4y^2<0$ to assure
     that $x^2+xy+y^2 >0$ for all $x$, given that $y\neq 0$ (since $y^2-4y^2 =
     0$ if $y=0$). In the case that $y=0$, $x^2+xy+y^2 = x^2$ is still positive given
     that $x\neq 0$.          
     \end{proof}
     \pagebreak
    \item For which numbers $\alpha $ is it true that $x^2+\alpha xy+y^2 >0$
    whenever $x$ and $y$ are not both $0$.
    \begin{proof} From the previous exercise, $b^2-4c = (\alpha y)^2-4y^2 <0$
    will assure that $x^2+\alpha xy+y^2 >0$, given that both $x$ and $y$ are not
    $0$. 

    We just solve for the values of $\alpha$ that will assure $(\alpha y)^2
    -4y^2$ is less than zero.
    \begin{align*}
        (\alpha y)^2 -4y^2 &<0, \\
        \alpha^2 y^2 &< 4y^2, \\
        \alpha ^2 &< 4, \\
        |\alpha | &< 2. \\
    \end{align*}
    so the values of $\alpha$ that will make $x^2+\alpha xy +y^2 >0$ are
    \begin{equation*}
        -2 <\alpha < 2.
    \end{equation*}
    \end{proof}
    \item Find the smallest possible value of $x^2+bx+c$ and of $ax^2+bx+c$, for
    $a>0$.
    \begin{proof} By completing the square, we can find the minimum value.
        \begin{equation*}
            x^2+bx+c = \left( x+\frac{b}{2}\right)^2 +c-\frac{b^2}{4},
        \end{equation*}
        and the minimum $c-\dfrac{b^2}{4}$ is achieved if $x= -\dfrac{b}{2}$.

        For $ax^2+bx+c$ where $a>0$,
        \begin{align*}
            ax^2+bx+c &= a\left(x^2+\frac{b}{a}x\right)+c \\
            &= a\left( x+ \frac{b}{2a}\right)^2 + c- \frac{b^2}{4a},
        \end{align*}
        and the minimum $c-\dfrac{b^2}{4a}$ is achieved if $x= -\dfrac{b}{2a}$.
    \end{proof}
\end{enumerate}
\end{exercise}
\pagebreak
\begin{exercise}[\textbf{20}]Prove that if
     \begin{equation*}
         |x-x_0|<\frac{\varepsilon}{2} \quad \text{ and } \quad |y-y_0|< \frac{\varepsilon}{2},
     \end{equation*}
     then
     \begin{align*}
         |(x+y) - (x_0 +y_0)| &< \varepsilon , \\
         |(x-y) - (x_0 - y_0)| &< \varepsilon .     
     \end{align*}
     \begin{proof}Assume the premise. Therefore, their sum would be 
          \begin{equation*}
              |x-x_0|+|y-y_0| < \varepsilon,
          \end{equation*}
          By the \emph{Triange Inequality} we get
          \begin{align*}
              |(x+y)-(x_0+y_0)|&=|(x-x_0)+(y-y_0) | \\
              &\leq |x-x_0|+|y-y_0| \\
              &< \varepsilon.
          \end{align*}
          Notice also that,
          \begin{equation*}
            |x-x_0|+|y-y_0|=|x-x_0|+|y_0 -y| < \varepsilon,
          \end{equation*}
          and by the \emph{Triangle Inequality},
          \begin{align*}
              |(x-y)-(x_0 -y_0) | &=|(x-x_0)-(y_0-y) | \\
              &\leq |x-x_0|+|y_0 -y| \\
              &< \varepsilon.
          \end{align*}
     \end{proof}
\end{exercise}