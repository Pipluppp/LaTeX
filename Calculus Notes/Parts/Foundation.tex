\section{Foundation}
\subsection{Sets}

\begin{defi}[Set]
A \emph{set} is a collection of stuff, without regards to order. Elements in a set are only counted once. For example, if $a = 2, b = c = 1$, then $A = \{a, b, c\}$ has only two members. We write $x\in X$ if $x$ is a member of the set $X$.
\end{defi}

\begin{eg}
Common sets and the symbols used to denote them:
\begin{itemize}
    \item $\N = \{1, 2, 3, \cdots \}$ is the natural numbers
    \item $\N_0 = \{0, 1, 2, \cdots \}$ is the natural numbers with $0$
    \item $\Z = \{\cdots, -2, -1, 0, 1, 2, \cdots \}$ is the integers
    \item $\Q = \{\frac{a}{b}: a, b\in \Z, b \not= 0\}$ is the rational numbers
    \item $\R$ is the real numbers
    \item $\C$ is the complex numbers
\end{itemize}
It is still debated whether $0$ is a natural number. Those who believe that $0$ is a natural number usually write $\N$ for $\{0, 1, 2, \cdots\}$, and $\N^+$ for the positive natural numbers. However, most of the time, it doesn't matter, and when it does, you should specify it explicitly.
\end{eg}
\begin{defi}[Equality of sets]
$A$ is equal to $B$, written as $A = B$, if
\[
    (\forall x)\,x\in A \Leftrightarrow x\in B,
\]
i.e.\ two sets are equal if they have the same elements.
\end{defi}

\begin{defi}[Subsets]
$A$ is a \emph{subset} of $B$, written as $A\subseteq B$ or $A\subset B$, if all elements in $A$ are in $B$. i.e.
\[
    (\forall x)\,x\in A\Rightarrow x\in B.
\]
\end{defi}

\begin{thm}
$(A=B)\Leftrightarrow (A\subseteq B \text{ and }B\subseteq A)$
\end{thm}

Suppose $X$ is a set and $P$ is the property of some elements in $x$, we can write a set $\{x\in X:P(x)\}$ for the subset of $x$ comprising of the elements for which $P(x)$ is true. e.g.\ $\{n\in \N : n \text{ is prime}\}$ is the set of all primes.

\begin{defi}[Intersection, union, set difference, symmetric difference and power set]
Given two sets $A$ and $B$, we define the following:
\begin{itemize}
    \item Intersection: $A\cap B = \{x:x\in A \text{ and } x\in B\}$
    \item Union: $A\cup B = \{x:x\in A\text{ or }x\in B\}$
    \item Set difference: $A\setminus B = \{x\in A: x\not\in B\}$
    \item Symmetric difference: $A\Delta B = \{x: x\in A\text{ xor } x\in B\}$, i.e.\ the elements in exactly one of the two sets
    \item Power set: $\mathcal{P}(A) = \{ X : X\subseteq A\}$, i.e.\ the set of all subsets
\end{itemize}
\end{defi}
\begin{defi}[Ordered pair]
An \emph{ordered pair} $(a, b)$ is a pair of two items in which order matters. Formally, it is defined as $\{\{a\}, \{a, b\}\}$. We have $(a, b) = (a', b')$ iff $a = a'$ and $b = b'$.
\end{defi}

\begin{defi}[Cartesian product]
Given two sets $A, B$, the \emph{Cartesian product} of $A$ and $B$ is $A\times B = \{(a, b):a\in A, b\in B\}$. This can be extended to $n$ products, e.g.\ $\R^3 = \R\times\R\times\R = \{(x,y,z): x, y, z\in \R\}$ (which is officially $\{(x, (y, z)): x, y, z\in \R\}$).
\end{defi}

\subsection{Relations}
\begin{defi}[Relation]
  A \emph{relation} $R$ on $A$ specifies that some elements of $A$ are related to some others. Formally, a relation is a subset $R\subseteq A\times A$. We write $aRb$ iff $(a, b)\in R$.
\end{defi}

\begin{eg}
  The following are examples of relations on natural numbers:
  \begin{enumerate}
    \item $aRb$ iff $a$ and $b$ have the same final digit. e.g.\ $(37)R(57)$.
    \item $aRb$ iff $a$ divides $b$. e.g.\ $2R6$ and $2\not \!\!R 7$.
    \item $aRb$ iff $a\not= b$.
    \item $aRb$ iff $a = b = 1$.
    \item $aRb$ iff $|a - b|\leq 3$.
    \item $aRb$ iff either $a, b\geq 5$ or $a, b\leq 4$.
  \end{enumerate}
\end{eg}

Again, we wish to classify different relations.
\begin{defi}[Reflexive relation]
  A relation $R$ is \emph{reflexive} if
  \[
    (\forall a)\,aRa.
  \]
\end{defi}

\begin{defi}[Symmetric relation]
  A relation $R$ is \emph{symmetric} iff
  \[
    (\forall a, b)\,aRb\Leftrightarrow bRa.
  \]
\end{defi}

\begin{defi}[Transitive relation]
  A relation $R$ is \emph{transitive} iff
  \[
    (\forall a, b, c)\,aRb\wedge bRc \Rightarrow aRc.
  \]
\end{defi}

\begin{eg}
  With regards to the examples above,
  \begin{center}
    \begin{tabular}{lcccccc}
      \toprule
      Examples & (i) & (ii) & (iii) & (iv) & (v) & (vi) \\
      \midrule
      Reflexive & \checkmark & \checkmark & $\times$ & $\times$ & \checkmark & \checkmark \\
      Symmetric & \checkmark & $\times$ & \checkmark & \checkmark & \checkmark & \checkmark \\
      Transitive & \checkmark & \checkmark & $\times$ & \checkmark & $\times$ & \checkmark \\
      \bottomrule
    \end{tabular}
  \end{center}
\end{eg}

\begin{defi}[Equivalence relation]
  A relation is an \emph{equivalence relation} if it is reflexive, symmetric and transitive. e.g.\ (i) and (vi) in the above examples are equivalence relations.
\end{defi}
If it is an equivalence relation, we usually write $\sim$ instead of $R$. As the name suggests, equivalence relations are used to describe relations that are similar to equality. For example, if we want to represent rational numbers as a pair of integers, we might have an equivalence relation defined by $(n, m)\sim (p, q)$ iff $nq = mp$, such that two pairs are equivalent if they represent the same rational number.

\begin{eg}
  If we consider a deck of cards, define two cards to be related if they have the same suite.
\end{eg}

As mentioned, we like to think of things related by $\sim$ as equal. Hence we want to identify all ``equal'' things together and form one new object.
\begin{defi}[Equivalence class]
  If $\sim$ is an equivalence relation, then the \emph{equivalence class} $[x]$ is the set of all elements that are related via $\sim$ to $x$.
\end{defi}

\begin{eg}
  In the cards example, $[8\heartsuit]$ is the set of all hearts.
\end{eg}

\begin{defi}[Partition of set]
  A \emph{partition} of a set $X$ is a collection of subsets $A_\alpha$ of $X$ such that each element of $X$ is in exactly one of $A_\alpha$.
\end{defi}

\subsection{Functions}
\begin{defi}[Function/map]
    A \emph{function} (or \emph{map}) $f: A\to B$ is a ``rule'' that assigns, for each $a\in A$, precisely one element $f(a)\in B$. We can write $a\mapsto f(a)$. $A$ and $B$ are called the \emph{domain} and \emph{co-domain} respectively.
  \end{defi}
  If we wish to be very formal, we can define a function to be a subset $f\subseteq A\times B$ such that for any $a\in A$, there exists a unique $b\in B$ such that $(a, b)\in f$. We then think of $(a, b) \in f$ as saying $f(a) = b$. However, while this might act as a formal definition of a function, it is a terrible way to think about functions.
  
  \begin{eg}
    $x^2: \R \to \R$ is a function that sends $x$ to $x^2$. $\frac{1}{x}:\R\to\R$ is not a function since $f(0)$ is not defined. $\pm x: \R\to\R$ is also not a function since it is multi-valued.
  \end{eg}
  
  It is often helpful to categorize functions into different categories.
  \begin{defi}[Injective function]
    A function $f: X \to Y$ is \emph{injective} if it hits everything at most once, i.e.
    \[
      (\forall x, y\in X)\,f(x) = f(y)\Rightarrow x = y.
    \]
  \end{defi}
  
  \begin{defi}[Surjective function]
    A function $f: X \to Y$ is \emph{surjective} if it hits everything at least once, i.e.
    \[
      (\forall y\in Y)(\exists x\in X)\,f(x) = y
    \]
  \end{defi}
  
  \begin{eg}
    $f: \R \to\R^+\cup\{0\}$ with $x \mapsto x^2$ is surjective but not injective.
  \end{eg}
  
  \begin{defi}[Bijective function]
    A function is \emph{bijective} if it is both injective and surjective. i.e.\ it hits everything exactly once.
  \end{defi}
  
  \begin{defi}[Permutation]
    A \emph{permutation} of $A$ is a bijection $A\to A$.
  \end{defi}
  
  \begin{defi}[Image of function]
    If $f: A\to B$ and $U\subseteq A$, then $f(U) = \{f(u):u\in U\}$. $f(A)$ is
    the \emph{image} of $A$.
  \end{defi}
  By definition, $f$ is surjective iff $f(A) = B$.

\subsection{Function Combinations and Compositions}
New functions can be built from old ones in many ways. Typically the old
functions are common, simple functions that are put together to construct a more
complex function that models some observed phenomenon

Starting with the notions of taking sums, differences, products, and quotients
of functions whose domains have nonempty intersections.

\begin{defi}[Combination of functions]
     Let $f:X \rightarrow \R$ and $g:Y \rightarrow \R$ be functions such that
     $X\cap Y \neq \emptyset$. Define functions $f+g$,$f-g$,$fg$, $X\cap Y \rightarrow \R$ by 
     \begin{equation*}
         (f+g)(x)=f(x)+g(x), \qquad (f-g)(x)=f(x)-g(x), \qquad (fg)(x)=f(x)g(x)
     \end{equation*}
     for all $x \in X\cap Y$.
     Let $Z=\{x \in X \cap Y:g(x)\neq 0\}$, and define $\tfrac{f}{g}:Z \rightarrow \R$ by 
     \begin{equation*}
         \left( \dfrac{f}{g}\right)(x) = \dfrac{f(x)}{g(x)}
     \end{equation*}
\end{defi}
\begin{defi}[Composition of functions]
    The \emph{composition} of two functions is a function you get by applying
    one after another. In particular, if $f: X \rightarrow Y$ and $G:
    Y\rightarrow Z$, then $g\circ f: X \rightarrow Z$ is defined by $g\circ f(x)
    = g(f(x))$. Note that function composition is associative.
  \end{defi}
  \begin{defi}[Constant functions]
    For each $a \in \R$ there is a function $C_a : \R \rightarrow \R$ such that
    for all $x \in \R$, $C_a(x) = a$. In other words, the output of the function does
    not depend on the input: whatever we put in, the same value a will come out.
    The graph of such a function is the horizontal line $y = a$. Such functions
    are called \emph{constant}.
  \end{defi}
  \begin{defi}[Identity functions]
       $I:\R \rightarrow\R$ by $I(x)=x$. The graph of the identity
       function is the straight line $y = x$.
  \end{defi}
  Recall that the identity function is so-called because it is an identity
element for the operation of function composition: that is, for any function $f :
\R \rightarrow \R$ we have $I \circ f$ = $f \circ I = f$.
\begin{eg}
     Let $m, b \in R$, and consider the function $L : R \rightarrow R$ by $x \mapsto
    mx + b$. Then $L$ is built up out of constant functions and the identity
    function by addition and multiplication: $L = C_m \cdot I + C_b$.
\end{eg}
\begin{eg}
    Let $n \in Z^+$. The function $m_n : x\mapsto x^n$ is built up out of the
    identity function by repreated multiplication: $m_n = $ $I \cdot I \cdots I$
    ($n$ $I$’s altogether).
\end{eg}
\begin{defi}
    The general name for a function $f : \R \rightarrow \R$ which is built up out of the
    identity function and the constant functions by finitely many additions and
    multiplications is a \emph{polynomial}. In other words, every \emph{polynomial function}
    is of the form
    \begin{equation*}
        f:x \mapsto a_nx^n+\cdots +a_1x+a_0
    \end{equation*}
    for some constants $a_0,\ldots,a_n \in \R$ 
\end{defi}
\subsection{Triangle Inequalities}
The following are essential and useful in proving some theorems, especially in
Analysis.
\begin{prop}
   For any $t \in \R$, 
   \begin{equation*}
     -\abs{t} \leq t \leq \abs{t}
   \end{equation*}
\end{prop}
\begin{prop}
  For any $x,y \in \R$ such that $x,y$ are nonnegative
   \begin{equation*}
     x^2 < y^2 \Rightarrow x < y
   \end{equation*}
\end{prop}
\begin{thm}[Triangle Inequality]
  For any $x,y \in \R$,
  \begin{equation*}
     \abs{x+y} \leq \abs{x}+\abs{y}
  \end{equation*}
     This proof is from \emph{Spivak Calculus} motivated by the observation that
      \begin{equation*}
       \abs{x} = \sqrt{x^2}
     \end{equation*}
     where $\sqrt{x}$ denotes the \emph{positive} square root of $x$, defined
     only when $x \geq 0$.
     \begin{proof}
        \begin{align*}
          (\abs{x+y})^2=(x+y)^2 &= x^2+2xy+x^2 \\
          &\leq x^2+2\abs{x}\cdot \abs{y}+y^2\\
          &= \abs{x}^2+2\abs{x}\cdot\abs{y}+\abs{y}^2\\
          &= (\abs{x}+\abs{y})^2.
        \end{align*}
        From this we can conclude that $\abs{x+y} < \abs{x} \abs{y}$ because $x^2 <
        y^2$ implies $x < y$, provided that $x$ and $y$ are both nonnegative.
     \end{proof}
     \begin{proof}[Alternative proof]
        Let $x,y \in \R$. We have $-\abs{x} \leq x \leq \abs{x}$ and $-\abs{y}
        \leq y \leq \abs{y}$, and adding these two yields
        \begin{equation*}
          -(\abs{x}+\abs{y}) \leq x+y \leq \abs{x}+\abs{y.}
        \end{equation*}
        Therefore  $\abs{x+y} \leq \abs{x}+\abs{y}$.
     \end{proof}
\end{thm}
\begin{cor}
  For any $x, y \in \R$.
  \begin{equation*}
    \abs{x-y} \leq \abs{x}+\abs{y}
  \end{equation*}
  \begin{proof}
    This follows immediately from the proposition above and the fact that $-\abs{t} =
    \abs{t}$ for any $t \in \R$.
    \begin{equation*}
      \abs{x-y} = \abs{x+(-y)} \leq \abs{x}+\abs{-y}= \abs{x}+\abs{y}.
    \end{equation*}
  \end{proof}
\end{cor}
\newpage
\begin{thm}[Reverse Triangle Inequality]
  For any $x, y \in \R$,
  \begin{equation*}
    \left\lvert \abs{x}-\abs{y}\right\rvert  \leq \abs{x-y}
  \end{equation*}
  \begin{proof}
     Let $x, y \in \R$. Using the \emph{Triangle Inequality}, we obtain the
     Inequalities
     \begin{equation*}
       \abs{x} =\abs{(x-y)+y} \leq \abs{x-y}+\abs{y}
     \end{equation*}
     and
     \begin{equation*}
       \abs{y} = \abs{(y-x)+x} \leq \abs{y-x}+\abs{x} = \abs{x-y}+\abs{x},
     \end{equation*}
     which become
     \begin{equation*}
       \abs{x}-\abs{y} \leq\abs{x-y} \text{ and } \abs{x}-\abs{y} \geq -\abs{x-y},
     \end{equation*}
     so that
     \begin{equation*}
       -\abs{x-y} \leq \abs{x}-\abs{y} \leq \abs{x-y},
     \end{equation*}
     and therefore $\left\lvert \abs{x}-\abs{y}\right\rvert \leq \abs{x-y}$.
  \end{proof}
\end{thm}
The \emph{Triangle Inequality} has the generalization, for any
$x_1,x_2,\ldots,x_n \in \R$,
\begin{equation*}
  \abs{x_1+x_2+\cdots+x_n} \leq \abs{x_1}+\abs{x_2}+\cdots+\abs{x_n},
\end{equation*}
or in sigma notation
\begin{equation*}
  \left\lvert \sum_{k=1}^n x_k\right\rvert \leq \sum_{k=1}^n\abs{x_k}.
\end{equation*}

 