\section{Direct Proof and Proof by Contrapostive}
We will now start the main gist of the notes. For a given true mathematical
statement, how can we show that it is true? 
\begin{defi}[Axiom]
     A true mathematical statement whose truth is accepted without proof.
\end{defi}
\begin{defi}[Theorem]
    A true mathematical statement whose truth can be verified 
\end{defi}
\begin{defi}[Corrolary]
    A mathematical result that can be deduced from, and is thereby a
    consequence of, some earlier result
\end{defi}
\begin{defi}[Lemma]
    A mathematical result that is useful in establishing the truth of some other
    result
\end{defi}

Most theorems (or results) are stated as implications. We now begin our study of
proofs of such mathematical statements.

\subsection{Trivial and Vacuous Proofs}
In nearly all of the implications $P \Rightarrow Q$ that we will encounter, $P$
and $Q$ are open sentences; that is, we will actually be considering $P(x)
\Rightarrow Q(x)$ or $P(n) \Rightarrow Q(n)$ or some related implication,
depending on which variable is being used. The variables $x$ or $n$ (or some other
symbols) are used to represent elements of some set $S$ being discussed, that is,
$S$ is the domain of the variable.
\begin{defi}[Trivial Proof]
    Let $P(x)$ and $Q(x)$ be open sentences over a domain $S$. Then $\forall x
    \in S, P(x) \Rightarrow Q(x)$ is a true statement if it can be shown that
    $Q(x)$ is true for all $x \in S$ (regardless of the truth value of $P(x)$,
    according to the truth table for implication.
\end{defi}
\begin{eg}
     Let $x \in \R$. If $x<0$, then $x^2+1>0$.
     \begin{proof}
          Since $x^2 \geq 0$ for each real number $x$. it follows that 
          \begin{equation*}
              x^2+1>x^2 \geq 0
          \end{equation*}
          Hence, $x^2+1>0$.
     \end{proof} 
\end{eg}
Since  we verified the truth of $Q(x)$ for every $x \in \R$, it follows that
$P(x) \Rightarrow Q(x)$ is true for all $x \in \R$. The proof did not depend on
$P(x)$, in fact we could have replaced it by any hypothesis and the result would
still be true.  
\begin{defi}[Vacuous Proof]
    Let $P(x)$ and $Q(x)$ be open sentences over a domain $S$. Then $\forall x
    \in S$, $P(x) \Rightarrow Q(x)$ is a true statement if it can be shown that
    $P(x)$ is false for all $x \in S$ (regardless of the truth value of $Q(x)$),
    according to the truth table for implication.
\end{defi}
\begin{eg}
     Let $x\in\R$. If $x^2-2x+2\leq0$,  then $x^2\geq8$.
     \begin{proof}
          First observe that
          \begin{equation*}
              x^2-2x+1 = (x-1)^2 \geq 0
          \end{equation*}
          Therefore, $x^2-2x+2 = (x-1)^2+1\geq1>0$.Thus, $x^2-2x+2 \leq 0$ is false for 
          all $x \in \R$ and the implication is true.
     \end{proof}
\end{eg}
Even though the trivial and vacuous proofs are rarely encountered in mathematics,
they are important reminders of the truth table for implication.

\subsection{Direct Proofs}
Typically, when we are discussing an implication $P(x) \Rightarrow Q(x)$ over a
domain $S$, there is some connection between $P(x)$ and $Q(x)$. That is, the truth
value of $Q(x)$ for a particular $x \in S$ often depends on the truth value of $P(x)$
for that same element $x$, or the truth value of $P(x)$ depends on the truth value
of $Q(x)$.
\begin{defi}[Direct Proof]
    Let $P(x)$ and $Q(x)$ be open sentences over a domain $S$. To give a
    \emph{direct proof} of $P(x) \Rightarrow Q(x)$ for all $x \in S$, we assume
    that $P(x)$ is true for an arbitrary element $x \in S$ and show that $Q(x)$
    must be true for this element $x$.
\end{defi}
If $P(x)$ is true, then $P(x) \Rightarrow Q(x)$ follows vacuously. Hence, we need
only be concerned with showing that $P(x) \Rightarrow Q(x)$ is true for all $x
\in S$ for which $P(x)$ is true.

We define the set of all even integers as
\begin{equation*}
    E = \{ 2k : k \in \Z\} = \{\ldots,-4,-2,0,2,4,\ldots \}.
\end{equation*}
and the set if all odd integers as
\begin{equation*}
    O = \{ 2k+1 : k \in \Z\} = \{\ldots,-5,-3,-1,1,3,5\ldots \}.
\end{equation*}
\begin{eg}
    If $n$ is an even integer, then $-5n-3$ is an odd integer.  
    \begin{proof}
         Let $n$ be an even integer. Then $n=2x$, where $x$ is an integer. Therefore,
         \begin{equation*}
                   -5n-3 = -5(2x)-3=-10x-3=-10x-4+1=2(-5x-2)+1.
         \end{equation*} 
         Since $-5x-2$ is an integer, $-5n-3$ is an odd integer.
    \end{proof}
\end{eg}

\begin{eg}
    If $n$ is an odd integer, then $4n^3+2n-1$ is odd.
    \begin{proof}
         Assume that $n$ is odd. Then $n=2y+1$ for some integer $y$. Therefore,
         \begin{align*}
             4n^3+2n-1  &= 4(2y+1)^3+2(2y+1)-1 \\
             &= 4(8y^3+12y^2+6y+1)+4y+2-1 \\
             &= 32y^3+48y^2+28y+5 \\
             &= 2(16y^3+24y^2+14y+2)+1.
         \end{align*}
         Since  $16y^3+24y^2+14y+2$ is an integer, $4n^3+2n-1$ is odd.
    \end{proof}
\end{eg}

Although the direct proof that we gave is correct, this is not the
\emph{desired} proof. Indeed, had we observed that  
\begin{equation*}
    4n^3+2n-1 = 2(2n^3+n-1)+1
\end{equation*}
and that $2n^3+n-1 \in \Z$, we could have concluded immediately that
$4n^3+2n-1$ is odd for every integer $n$. Hence, a trivial proof could be
given and, in fact, preferred.  

\subsection{Proof by Contrapositive}
\begin{defi}[Proof by Contrapostive]
    For statements $P$ and $Q$, the contrapositive of the implication $P
    \Rightarrow Q$ is the implication $(\sim Q) \Rightarrow (\sim P)$
\end{defi}
\begin{thm}
    For every two statements $P$ and $Q$, the implication $P \Rightarrow Q$ and
    its contrapositive are logically equivalent; that is,
    \begin{equation*}
        P\Rightarrow Q \equiv (\sim Q) \Rightarrow (\sim P).
    \end{equation*}
    \begin{center}
        \begin{tabular}{cccccccc}
          \toprule
          $P$ & $Q$ &$P \Rightarrow Q$ & $ \sim Q$ & $ \sim P$ & 
          $(\sim Q) \Rightarrow (\sim P)$ \\
          \midrule
          T & T & \textbf{T} & F & F & \textbf{T} \\
          T & F & \textbf{F} & T & F & \textbf{F} \\
          F & T & \textbf{T} & F & T & \textbf{T} \\
          F & F & \textbf{T} & T & T & \textbf{T} \\
          \bottomrule
        \end{tabular}
      \end{center}
\end{thm}
\begin{eg}
     Let $x \in \Z$. If $5x-7$ is even, then $x$ is odd.
     \begin{proof}
        Assume that $x$ is even. Then $x = 2a$ for some integer $a$. So
        \begin{equation*}
            5x-7=5(2a)-7=10a-7=10a-8+1=2(5a-4)+1
        \end{equation*}
        Since $5a-4 \in \Z$, the integer $5x-7$ is odd.
     \end{proof}
\end{eg}
\begin{thm}
     Let $x \in \Z$. Then $x^2$ is even if and only if $x$ is even.
\end{thm}
\begin{proof}
     Assume that $x$ is even. Then $x=2a$ for some integer $a$. Therefore,
     \begin{equation*}
         x^2 = (2a)^2 = 4a^2=2(2a^2).
     \end{equation*}
     Because $2a^2 \in \Z$, the integer $x^2$ is even.

     For the converse, assume that $x$ is odd. So, $x=2b+1$, where $b \in \Z$.
     Then
     \begin{equation*}
         x^2 = (2b+1)^2 = 4b^2+4b+1 = 2(2b^2+2b)+1
     \end{equation*} 
     Since $2b^2+2b \in \Z$, $x^2$ is odd.
\end{proof}
Suppose that we have been successful in proving $P(x) \Rightarrow Q(x)$ for all
$x$ in some domain $S$ (by whatever method). We therefore know that for every $x
\in S$ for which the statement $P(x)$ is true, the statement $Q(x)$ is true.
Also, for any $x \in S$ for which the statement $Q(x)$ is false, the statement
$P(x)$ is false.

\begin{problem}
    Let $x \in \Z$. If $5x-7$ is odd, then $9x+2$ is even   
    \begin{lemma}
        Let $x \in \Z$. If $5x-7$ is odd, then $x$ is even
    \end{lemma}
    \begin{proof}
        Let $5x-7$ be an odd integer. By the \emph{Lemma}, the integer $x$ is
        even. Since $x$ is even, $x=2z$ for some integer $z$. Thus,
        \begin{equation*}
            9x+2=9(2z)+2=18z+2=2(9z+1).
        \end{equation*}
        Because $9z+1$ is an integer, $9x+2$ is even.
    \end{proof}
    \begin{proof}[Alternative Proof]
        Assume that $5x-7$ is odd. Then $5x-7 = 2n+1$ for some integer $n$.
        Observe that
        \begin{align*}
            9x+2&=(5x-7)+(4x+9)=2n+1+4x+9 \\[1.25ex]
            &= 2n+4x+10=2(n+2x+5).
        \end{align*}
        Because $n + 2x + 5$ is an integer, $9x + 2$ is even.
    \end{proof}
\end{problem}

\subsection{Proof by Cases}
\begin{defi}
    For a mathematical statement concerning an element $x \in S$,  If we can
    verify the truth of the statement for each property that $x$ may have, then
    we have a proof of the statement. Such a proof is then divided into parts
    called \emph{cases}, one case for each property that $x$ may possess or for
    each subset to which $x$ may belong
\end{defi}
For example, in a proof of $\forall n \in \Z, R(n),$ it might be convenient
to use a proof by cases whose proof is divided into the two cases
\begin{itemize}
        \item Case $1$: $n$ \textit{is even}
        \item Case $2$: $n$ \textit{is odd}
\end{itemize}
or in the case of $\forall x \in \R, P(x)$
\begin{itemize}
        \item Case $1$: $x=0$
        \item Case $2$: $x<0$
        \item Case $3$: $x>0$
\end{itemize}
we also might attempt to prove $\forall n \in \N, P(n)$ using the cases
\begin{itemize}
        \item Case $1$: $n=1$
        \item Case $2$: $n\geq 2$
\end{itemize}
Furthermore, for $S=\Z-\{0\}$, we might try to prove $\forall x,y \in S,
P(x,y)$ by using the cases.
\begin{itemize}
     \item Case $1$: $xy>0$
     \item Case $2$: $xy<0$
\end{itemize}
Case $1$ could, in fact, be divided into two subcases
\begin{equation*}
    Subcase\;1.1: x > 0 \text{ and } y > 0. \text{ and } Subcase\;1.2: x < 0 \text{ and } y < 0.
\end{equation*}
while Case $2$ could be divided into two subcases
\begin{equation*}
    Subcase\;2.1: x > 0 \text{ and } y < 0. \text{ and } Subcase\;2.2: x < 0 \text{ and } y > 0.
\end{equation*}
\begin{defi}[Same Parity]
    Two integers $x$ and $y$ are said to be \emph{of the same parity} if $x$ and
    $y$ are both even or are both odd
\end{defi}
\begin{defi}[Opposite Parity]
    The integers $x$ and $y$ are \emph{of opposite parity}
    if one of $x$ and $y$ is even and the other is odd.
\end{defi}
\begin{thm}
    Let $x+y \in \Z$. Then $x$ and $y$ are of the same parity if and only if
    $x+y$ is even.
    \begin{proof}
        $(\Rightarrow)$ First, assume that $x$ and $y$ are of the same parity. We consider two
        cases.
        \begin{itemize}
             \item Case $1$: $x$ and $y$ are even. Then $x = 2a$ and $y = 2b$
             for some integers $a$ and $b$. So, $x + y = 2a + 2b = 2(a + b)$.
             Since $a + b \in Z$, the integer $x + y$ is even.
             \item Case $2$: $x$ and $y$ are odd. Then $x = 2a + 1$ and$ y = 2b + 1$,
             where $a, b \in Z$. Therefore
             \begin{equation*}
                x + y = (2a + 1) + (2b + 1) = 2a + 2b + 2 = 2(a + b + 1).
             \end{equation*}
             Since $a + b + 1$ is an integer, $x + y$ is even
        \end{itemize}
        $(\Leftarrow)$ For the converse, assume that $x$ and $y$ are of opposite parity. Again,
        we consider two cases. 
        \begin{itemize}
             \item Case $1$: $x$ is even and $y$ is odd. Then $x = 2a$ and $y = 2b + 1$,
             where $ a, b \in Z$. Then
             \begin{equation*}
                x + y = 2a + (2b + 1) = 2(a + b) + 1
             \end{equation*}
             Since $a + b \in \Z$, the integer $x + y$ is odd.
             \item Case $2$: $x$ is odd and $y$ is even. The proof is similar to
             the proof of the preceding case and is therefore omitted.
        \end{itemize}
    \end{proof}
\end{thm}
\begin{remark}
    Although there is always some concern when omitting steps or proofs, it
    should be clear that it is truly a waste of effort to give a proof of the
    case when $x$ is odd and $y$. However,  there is an alternative when the
    converse is considered.

    For the converse, assume that $x$ and $y$ are of opposite parity.\emph{
    Without loss of generality}, assume that $x$ is even and $y$ is odd. Then $x
    = 2a$ and $y = 2b + 1$, where $a, b \in Z$. Then
    \begin{equation*}
        x + y = 2a + (2b + 1) = 2(a + b) + 1
    \end{equation*}
    Since $a + b \in \Z$, the integer $x + y$ is odd.

    We used the phrase \textbf{\emph{without loss of generality}} to indicate
    that the proofs of the two situations are similar, so the proof of only one
    of these is needed.
\end{remark}
\begin{thm}
    Let $a$ and $b$ be integers. Then $ab$ is even if and only if $a$ is even or
    $b$ is even.
    \begin{proof}
        $(\Leftarrow)$ First, assume that $a$ is even or $b$ is even. Without loss of
        generality, let $a$ be even. Then $a = 2x$ for some integer $x$. Thus,
        $ab = (2x)b = 2(xb)$. Since $xb$ is an integer, $ab$ is even.

        $(\Rightarrow)$ For the converse, assume that $a$ is odd and $b$ is odd. Then $a = 2x +
        1$ and $b = 2y + 1$, where $x, y \in Z$. Hence,
        \begin{equation*}
            ab = (2x + 1)(2y + 1) = 4xy + 2x + 2y + 1 = 2(2xy + x + y) + 1.
        \end{equation*}
        Since $2xy + x + y$ is an integer, $ab$ is odd.
    \end{proof}
\end{thm}
\begin{result}
    Let $\mathcal{P} = \{A, B,C\}$ be a partition of the set $\Z$ of integers,
    where
    \begin{itemize}
         \item[] $A$ = \{$n : n = 2a$ where $a$ is an odd integer\}
         \item[] $B$ = \{$n:n=2b$ where $b$ is an even integer\} 
         \item[] $C$ = \{$n : n$ is an odd integer\}
    \end{itemize}
    If $x$ and $y$ are integers belonging to distinct elements of $\mathcal{P}$,
    then $x + y \in A \cup C$
    \begin{proof}
        (Consider three cases)
    \end{proof}
\end{result}

\newpage
\subsection{Proof Evaluations}
We now reverse the process, where we give the proof of some result, and we need
to find the result.
\begin{eg}
     Evaluate the proposed proof of the following result.
     \begin{center}
        If $m$ is an even integer and $n$ is an odd integer, then $3m + 5n$ is
        odd.
     \end{center}
     \begin{proof}
        Let $m$ be an even integer and $n$ an odd integer. Then $m = 2k$ and $n =
        2k + 1$, where $k \in \Z$. Therefore
        \begin{align*}
            3m+5n &= 3(2k)+5(2k+1) = 6k+10k+5 \\
            &= 16k+5 = 2(8k+2)+1.
        \end{align*}
        Since $8k+2$ is an integer, $3m+5n$ is odd.
     \end{proof}
     \begin{proof}[Proof Evaluation]
        There is a mistake in the second sentence of the proposed proof, where
        it is written that $m = 2k$ and $n = 2k + 1$, where $k \in \Z$.

        Since the same symbol $k$ is used for both $m$ and $n$, we have
        inadvertently added the assumption that $n = m + 1$. This is incorrect,
        as it was never stated that $n$ and $m$ must be consecutive integers. 

        In other words, we should write $m=2k$ and $n=2\ell+1$, where $k,\ell    \in
        \Z$.
     \end{proof}
\end{eg}

\section{More on Direct Proof and Proof by Contrapositive}

\subsection{Proofs on Divisibility of Integers}
\begin{defi}
     In general, for integers $a$ and $b$ with $a \neq 0$, we say that $a$
     divides $b$, written as
     \[a \mid b\]
     if there is an integer $c$ such that $b=ac$.
\end{defi}

Hence, if $n$ is an even integer, then $2 \mid n$; moreover, if $2$ divides some
integer $n$, then $n$ is even. That is, an integer $n$ is even if and only if $2
\mid n$.

Therefore, the \emph{Theorem},
\begin{center}
     For integers $a$ and $b$, $ab$ is even, if and only if $a$ or $b$ is even. 
\end{center}
can be rewritten as
\begin{center}
     For integers $a$ and $b$, $2 \mid ab$ if and only if $2 \mid a$ or $2 \mid b$.
\end{center}

If $a \mid b$, then we also say that $b$ is a \emph{multiple} of $a$ and that
$a$ is a \emph{divisor} of $b$. Thus, every even integer is a multiple of $2$.
If $a$ does not divide $b$, then we write $a \nmid b$. 
\medbreak
For example, $4 \mid 48$, since $4 \cdot 12 = 48$ and $-3 \mid 57$ since $57 =
(-3) \cdot (19)$. On the other hand, $4\nmid 66$ as there is no integer $c$ such
that $66 = 4c$   .
\subsection{Proofs of Congruence of Integers }
\begin{defi}
    For integers $a, b$ and $n \geq 2$, we say that $a$ is \textbf{congruent} to
    $b$ \textbf{modulo} $n$, written
    \[a \equiv b\text{ (mod $n$), if }n \mid (a-b)\]
\end{defi}
\subsection{Proofs on Real Numbers}
\subsection{Proofs on Sets}
\subsection{Fundamental Properties of Set Operations}
\subsection{Proofs on Cartesian Products of Sets}


