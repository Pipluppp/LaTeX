     \section{Logic}
\subsection{Statements}
\begin{defi}[Statement]
     A \emph{statement} is a declarative sentence or assertion that has a truth value, namely \emph{true} or \emph{false}.
\end{defi}
\begin{eg}
     The following are statements:
     \begin{enumerate}[label=(\roman*)]
          \item The integer $3$ is odd.
          \item The integer $57$ is prime.
     \end{enumerate}
\end{eg}
\begin{defi}[Open Sentence]
    An \emph{open sentence} is a declarative sentence that contains one
    or more variables, each variable representing a value in some prescribed set, called the
    \emph{domain} of the variable.
\end{defi}
\begin{eg}
     The following are open sentences with variable $x$ in some domain $S$.
     \begin{enumerate}
          \item $3x=4$
          \item $x^2+4x+3=0$
     \end{enumerate}
     An open sentence that contains a variable x is typically represented by $P(x)$, $Q(x)$ or
    $R(x)$. If $P(x)$ is an open sentence, where the domain of $x$ is $S$, then we say $P(x)$ is an open
    sentence over the domain $S$.
\end{eg}

\subsection{Negations}
\begin{defi}[Negation]
     The \emph{negation} of a statement $P$, denoted by $ \sim P$ is the statement
     \begin{center}
          \textbf{not} $P$
     \end{center}
\end{defi}
The truth table for $\sim P$ is 
\begin{center}
    \begin{tabular}{cccccccc}
      \toprule
      $P$ & $\sim P$  \\
      \midrule
      T & \textbf{F} \\
      T & \textbf{F} \\
      F & \textbf{T} \\
      F & \textbf{T} \\
      \bottomrule
    \end{tabular}
  \end{center}
\begin{eg}
    The negation of the statement
    \begin{equation*}
         P_1: \text{The integer $3$ is odd.}
    \end{equation*}
    is 
    \begin{equation*}
         \sim P_1:\text{The integer $3$ is even.}
    \end{equation*}
\end{eg}
\begin{eg}
     The negation of the statement          
    \begin{equation*}
          P_2:\text{The real number $r$ is at most $\sqrt{2}$} 
    \end{equation*}
    is 
    \begin{equation*}
         \sim P_2:\text{The real number $r$ is greater than $\sqrt{2}$}
    \end{equation*}
\end{eg}

\subsection{Disjunctions and Conjunctions}
\begin{defi}[Disjunction]
     The \emph{disjunction} of the statements $P$ and $Q$ is the statement
    \begin{center}
         $P$ \textbf{or} $Q$
    \end{center}
    denoted by $P \vee Q$.
\end{defi}
\begin{defi}
     The \emph{conjunction} of the statements $P$ and $Q$ is the statement
     \begin{center}
        $P$ \textbf{and} $Q$
     \end{center}
     denoted by $P \wedge Q$
\end{defi}
The truth table for $P \vee Q$ and $P \wedge Q$ is 
\begin{center}
    \begin{tabular}{cccccccc}
      \toprule
      $P$ & $Q$ &\quad &$P \vee Q$ &$P \wedge Q$ \\
      \midrule
      T & T & & \textbf{T} & \textbf{T} \\
      T & F & & \textbf{T} & \textbf{F} \\
      F & T & & \textbf{T} & \textbf{F} \\
      F & F & & \textbf{F} & \textbf{F} \\
      \bottomrule
    \end{tabular}
  \end{center}

\subsection{Implications}
\begin{defi}[Implication]
    For statements $P$ and $Q$, the \emph{implication}
    (or \emph{conditional}) is the statement
    \begin{center}
         \textbf{If} $P$, \textbf{then} $Q$.
    \end{center}
    denoted by $P \Rightarrow Q$. Which can also be expressed as
    \begin{center}
         $P$ \textbf{implies} $Q$.
    \end{center}
\end{defi}  
The truth table for $P \Rightarrow Q$ is 
\begin{center}
    \begin{tabular}{cccccccc}
      \toprule
      $P$ & $Q$ &\quad &$P \Rightarrow Q$  \\
      \midrule
      T & T & & \textbf{T} \\
      T & F & & \textbf{F} \\
      F & T & & \textbf{T} \\
      F & F & & \textbf{T} \\
      \bottomrule
    \end{tabular}
\end{center}   
In summary, $P \Rightarrow Q$ is false when $P$ is True and $Q$ is false.
\medbreak
The following have the same truth value

\begin{itemize}
     \item $\sim (P \Rightarrow Q)$ and P $\wedge \sim Q$
     \item $P \Rightarrow Q$ and $\sim P \vee Q$
\end{itemize}

The following are equivalent statements
\begin{itemize}
     \item If $P$, then $Q$
     \item $Q$ if $P$
     \item $P$ implies $Q$
     \item $P$ only if $Q$
     \item $P$ is sufficient for $Q$
     \item $Q$ is necessary for $P$
\end{itemize}

\begin{eg}
     A triangle is \emph{equilateral} if the lengths of its three sides 
     are the same, while \emph{isosceles} if any two of its three sides 
     are the same.

     For a triangle $T \in S$, let 
     \begin{center}
          $P(T)$: $T$ is equilateral \qquad and \qquad $Q(T)$: $T$ is isosceles
     \end{center}
     Consider the implication $P(T) \Rightarrow Q(T)$
     \begin{enumerate}[label=(\roman*)]
          \item For an equilateral triangle $T_1$, both $P(T_1)$ and $Q(T_1)$
          are true and so  $P(T_1) \Rightarrow Q(T_1)$ is also true.
          \item If $T_2$ is not an equilateral triangle, then $P(T_2)$ is a 
          false statement and so $P(T_2) \Rightarrow Q(T_2)$ is true regardless
          of the truth of $Q(T_2)$.
     \end{enumerate}
     We now state $P(T) \Rightarrow Q(T)$ in a variety of ways:
     \begin{itemize}
        \item If $T$ is an equilateral triangle, then $T$ is isosceles.
        \item  A triangle $T$ is an isosceles if $T$ is equilateral.
        \item A triangle $T$ is equilateral only if $T$ is isosceles.
        \item For a triangle $T$ to be isosceles, it is sufficient that $T$ be equilateral.
        \item For a triangle $T$ to be equilateral, it is necessary that $T$ be isosceles.
     \end{itemize}
\end{eg}

\subsection{Biconditionals}
\begin{defi}[Converse]
    For statements (or open sentences) $P$ and $Q$, the implication $Q \Rightarrow P$ 
    is called the \emph{converse} of $P \Rightarrow Q$.
\end{defi}
\begin{eg}
     For the statements
     \begin{center}
          $P_1$: $3$ is an odd integer \qquad and \qquad $P_2$: $57$ is prime.
     \end{center} 
     the converse of the implication
     \begin{center}
          $P_1 \Rightarrow P_2$: If $3$ is an odd integer, then $57$ is prime.
     \end{center}
     is the implication
     \begin{center}
          $P \Rightarrow Q$: If $57$ is prime, then $3$ is an odd integer. 
     \end{center}
\end{eg}
\begin{defi}[Biconditional]
    For statements (or open sentences) $P$ and $Q$, the conjunction
    \begin{equation*}
         (P \Rightarrow Q) \wedge (Q \Rightarrow P)
    \end{equation*}
    of the implication $P \Rightarrow Q$ and its converse is called the
    \emph{biconditional} of $P$ and $Q$, denoted by $P \Leftrightarrow Q$.
\end{defi}
Truth table of $P \Leftrightarrow Q$
\begin{center}
    \begin{tabular}{cccccccc}
      \toprule
      $P$ & $Q$ &\quad & $P\Rightarrow Q$ & $Q\Rightarrow P$ & 
      $(P\Rightarrow Q) \wedge (Q\Rightarrow P)$ & $P \Leftrightarrow Q$\\
      \midrule
      T & T & & T & T & T & \textbf{T} \\
      T & F & & F & T & F & \textbf{F} \\
      F & T & & T & F & F & \textbf{F} \\
      F & F & & T & T & T & \textbf{T} \\
      \bottomrule
    \end{tabular}
\end{center}
The biconditional $P \Leftarrow Q$ is often stated as 
\begin{center}
     $P$ \textbf{is equivalent to} $Q$.
\end{center}
or
\begin{center}
     $P$ \textbf{if and only if} $Q$. 
\end{center}
or as 
\begin{center}
     $P$ \textbf{is necessary and sufficient for} $Q$.
\end{center}
For statements $P$ and $Q$, it then follows that the biconditional \textit{“P if and only if Q”} 
is true only when $P$ and $Q$ have the same truth values.
\begin{eg}
     For the open sentences
     \begin{center}
          $P(T)$: $T$ is equilateral \qquad and \qquad $Q(T)$: $T$ is isosceles.
     \end{center}
     over the domain $S$ of all triangles, the converse of the implication
     \begin{center}
          $P(T) \Rightarrow Q(T)$: If $T$ is equilateral, then $T$ is isosceles.
     \end{center}
     is the implication
     \begin{center}
          $Q(T) \Rightarrow P(T)$: If $T$ is isosceles, then $T$ is equilateral.
     \end{center}
     We noted that $P(T) \Rightarrow Q(T $ is a true statement for all triangles $T$, while $Q(T) \Rightarrow P(T)$
    is a false statement when $T$ is an isosceles triangle that is not equilateral. On the other
    hand, the second implication becomes a true statement for all other triangles $T$. Therefore, the biconditional
     \begin{center}
          $P(T) \Leftrightarrow Q(T)$: $T$ is equilateral if and only if $T$ is isosceles.
     \end{center}
     is false for all triangles that are isosceles and not equilateral, while it is true for all other
    triangles $T$.
\end{eg}

\subsection{Tautologies and Contradictions}
\begin{defi}[Logical Connectives]
     A \emph{logical connective} is a symbol or word used to connect two or more sentences
     n a grammatically valid way, such that the value of the compound sentence produced 
     depends only on that of the original sentences and on the meaning of the connective.
     \begin{equation*}
          \sim,\:\wedge,\:\vee,\:\Rightarrow,\:\Leftrightarrow 
     \end{equation*}
\end{defi}
\begin{defi}[Compound statement]
    a \emph{compound statement}is a statement composedof one or more 
    given statements (called \emph{component statements} in this context) and
    at least one logical connective.
\end{defi}
\begin{eg}
     For a given statement $P$, its negation $\sim P$ is a compound statement.
\end{eg}
\begin{defi}[Tautology]
    A compound statement $S$ is called a \emph{tautology} if it is true for 
    all possible combinations of truth values of the component statements
\end{defi}
\begin{eg}
     The compound statement $P\;\vee \sim P$ is a tautology
     \begin{center}
        \begin{tabular}{cccccccc}
          \toprule
          $P$ & $\sim P$ &$P\;\vee \sim P$ \\
          \midrule
          T & F & \textbf{T} \\
          T & F & \textbf{T} \\
          F & T & \textbf{T} \\
          F & T & \textbf{T} \\
          \bottomrule
        \end{tabular}
      \end{center}
\end{eg}
\begin{eg}
    For statements $P$ and $Q$, the compound statement $(\sim Q) \vee (P \Rightarrow Q)$ 
    is a tautology, as is verified in the truth table.
    \begin{center}
        \begin{tabular}{cccccccc}
          \toprule
          $P$ & $Q$ &$\sim Q$ & $P \Rightarrow Q$& $(\sim Q) \vee (P \Rightarrow Q)$ \\
          \midrule
          T & T & F & T & \textbf{T} \\
          T & F & T & F & \textbf{T} \\
          F & T & F & T & \textbf{T} \\
          F & F & T & T & \textbf{T} \\
          \bottomrule
        \end{tabular}
      \end{center}
\end{eg}
\begin{defi}[Contradiction]
    A compound statement $S$ is called a \emph{contradiction} if it is false 
    for all possible combinations of truth values of the component statements.
\end{defi}
\begin{eg}
    The compound statement $P \: \wedge \sim P$ is a contradiction.
    \begin{center}
        \begin{tabular}{cccccccc}
          \toprule
          $P$ & $\sim P$ &$P\;\wedge \sim P$ \\
          \midrule
          T & F & \textbf{F} \\
          T & F & \textbf{F} \\
          F & T & \textbf{F} \\
          F & T & \textbf{F} \\
          \bottomrule
        \end{tabular}
      \end{center}
\end{eg}
\begin{eg}
    For statements $P$ and $Q$, the compound statement $(P \wedge Q) \wedge (Q \Rightarrow (\sim P))$ 
    is a contradiction, as is verified in the truth table.
    \begin{center}
        \begin{tabular}{cccccccc}
          \toprule
          $P$ & $Q$ &$\sim P$ & $P \wedge Q$ & $Q \Rightarrow (\sim P)$ & 
          $(P \wedge Q) \wedge (Q \Rightarrow (\sim P))$ \\
          \midrule
          T & T & F & T & F & \textbf{F} \\
          T & F & F & F & T & \textbf{F} \\
          F & T & T & F & T & \textbf{F} \\
          F & F & T & F & T & \textbf{F} \\
          \bottomrule
        \end{tabular}
      \end{center}
\end{eg}
\begin{defi}[Modus Ponens]
    For statements $P$ and $Q$, $(P \wedge (P \Rightarrow Q)) \Rightarrow Q$ 
    is a tautology. (This logical argument form is called \emph{modus ponens})
    \medbreak
    \begin{center}
        If $P$ is true and $P \Rightarrow Q$ is true, then $Q$ is true.
    \end{center}
    \begin{center}
        \begin{tabular}{cccccccc}
          \toprule
          $P$ & $Q$ &$P \Rightarrow Q$ & $P \wedge (P \Rightarrow Q)$ 
          & $(P \wedge (P \Rightarrow Q) \Rightarrow Q)$ & \\
          \midrule
          T & T & T & T & \textbf{T}  \\
          T & F & F & F & \textbf{T}  \\
          F & T & T & F & \textbf{T}  \\
          F & F & T & F & \textbf{T}  \\
          \bottomrule
        \end{tabular}
      \end{center}
\end{defi}
\begin{defi}[Syllogism]
    For statements $P,Q$, and $R$ show that 
    $((P \Rightarrow Q) \wedge (Q \Rightarrow R)) \Rightarrow R)$ 
    is a tautology. (This logical argument form is called \emph{syllogism})
    \medbreak
    \begin{center}
        If $P \Rightarrow Q$ and $Q \Rightarrow R$, then $P \Rightarrow R$
    \end{center}
\end{defi}
\subsection{Logical Equivalence}
\begin{defi}[Logical Equivalence]
    Let $R$ and $S$ be two compound statements involving the 
    same component statements $P$ and $Q$. Then $R$ and $S$ are 
    called \emph{logically equivalent} if $R$ and $S$ have the same truth
     values for all combinations of truth values of their component statements
     denoted by
     \begin{equation*}
          R \equiv S
     \end{equation*}
\end{defi}
\begin{eg}
    The statements $P \implies Q$ and $(\sim P) \vee Q$ are 
    logically equivalent,in other words 
    $P \implies Q \equiv (\sim P) \vee Q$.
    \begin{center}
        \begin{tabular}{cccccccc}
          \toprule
          $P$ & $Q$ &$\sim P$ & $P \Rightarrow Q$ 
          & $(\sim P) \vee Q$ & \\
          \midrule
          T & T & F & \textbf{T} & \textbf{T}  \\
          T & F & F & \textbf{F} & \textbf{F}  \\
          F & T & T & \textbf{T} & \textbf{T}  \\
          F & F & T & \textbf{T} & \textbf{T}  \\
          \bottomrule
        \end{tabular}
      \end{center}
\end{eg}
Logical equivalence guarantess the truth of the other statement.
If we can establish the truth of the statement $(\sim P) \vee Q$,
 then the logical equivalence of $P \Rightarrow Q$ and $(\sim P) \vee Q$ 
 guarantees $P \Rightarrow Q$ is also true.
\begin{eg}
     For the following statement
     \begin{center}
          \textit{If you earn an $A$ on the final exam, 
          then you will receive an $A$ for the final grade.}
     \end{center}
     We need only know that the student did not receive an A on 
     he final exam or the student received an $A$ as a final grade 
     to see that the instructor kept her promise. 
\end{eg}
\begin{thm}
     Let $P$ and $Q$ be two statements. Then
     \begin{equation*}
        P \Rightarrow Q \:\text{and}\: (\sim P) \vee Q
     \end{equation*}
     are logically equivalent.
\end{thm}
For the following statements
\begin{center}
     $Q \Rightarrow P$ is written as "$P$ if $Q$"
\end{center}
\begin{center}
    $P \Rightarrow Q$ is written as "$P$ only if $Q$".
\end{center}
their conjunction can be written as "$P$ if $Q$ and $P$ only if
 $Q$", or more simply
 \begin{center}
    $P$ if and only if $Q$
\end{center}

\subsection{Fundamental Properties of Logical Equivalence}
\begin{thm}
    For statements $P$, $Q$, and $R$.
    \begin{enumerate}
         \item \emph{Commutative Laws}
         \begin{itemize}
              \item $P \vee Q \equiv Q \vee P$
              \item $P \wedge Q \equiv Q \wedge P$
         \end{itemize}
         \item \emph{Associative Laws}
         \begin{itemize}
              \item $P \vee (Q \vee R) \equiv (P \vee Q) \vee R$
              \item $P \wedge (Q \wedge R) \equiv (P \wedge Q) \wedge R$
         \end{itemize}
         \item \emph{Distributive laws}
         \begin{itemize}
              \item $P \vee (Q \wedge R) \equiv (P \vee Q) \wedge (P \vee R)$
              \item $P \wedge (Q \vee R) \equiv (P \wedge Q) \vee (P \wedge R)$
         \end{itemize}
         \item \emph{De Morgan's Laws}
         \begin{itemize}
            \item $\sim (P \vee Q) \equiv (\sim P) \wedge (\sim Q)$
            \item $\sim (P \wedge Q) \equiv (\sim P) \vee (\sim Q)$
         \end{itemize}
    \end{enumerate}
\end{thm}
\begin{thm}
    For statements $P$ and $Q$
    \begin{itemize}
       \item $\sim(P \Rightarrow Q) \equiv P \wedge (\sim Q)$
       \item $ \sim (P \Rightarrow Q) \equiv (P\wedge (\sim Q))\vee(Q\wedge (\sim P))$
    \end{itemize}
\end{thm}

\begin{proof} We make use of the De Morgan's Laws
    \begin{align*}
        (P \Rightarrow Q) &\equiv\:\sim((\sim P) \vee Q)\\[1.25ex]
        & \equiv (\sim (\sim P)) \wedge (\sim Q) \\[1.25ex]
        & \equiv P \wedge \sim Q
    \end{align*}
\end{proof}
\begin{proof} We make use of the De Morgan's Laws
    \begin{align*}
        \sim (P \Leftarrow Q) &\equiv \sim ((P \Rightarrow Q) \wedge
        (Q \Rightarrow P)) \\[1.25ex]
        &\equiv (\sim (P \Rightarrow Q)) \vee (\sim (Q \Rightarrow P)) \\[1.25ex] 
        &\equiv (P \wedge (\sim Q)) \vee (Q \wedge (\sim P))
    \end{align*}
\end{proof}

\subsection{Quantified Statements}
If $P(x)$ is an open sentence over a domain $S$, then $P(x)$ 
is a statement for each $x \in S$. \emph{Quantification} is a method 
to convert open sentences to a \emph{quantified statement}.
\begin{defi}[Universal Quantifier]
    Adding the phrase \textit{"For every"} $x \in S$ to $P(x)$, where 
    \textit{"for every"} is referred as universal quantifier 
    denoted by $\forall$
    \begin{equation*}
        \forall x \in S,\: P(x).
    \end{equation*}
    in words
    \begin{equation*}
         \text{For every } x \in S,\: P(x)
    \end{equation*}
    The quantified statement is true if $P(x)$ is true for 
    every $x\in S$; while false if for at least one element
     $x\in S$, $P(x)$ is false.
\end{defi}
\begin{defi}[Existential Quantifier]
    Each of the phrases \textit{“there exists”}, \textit{“there is"}, 
    \textit{“for some”}, and \textit{“for at least one”} is
    referred to as an \emph{existential quantifier} and is denoted by 
    the symbol $\exists$. The quantified statement
    \begin{equation*}
         \exists x \in S, P(x)
    \end{equation*}
    can be expressed in words by
    \begin{center}
         There exists $x \in S$ such that $P(x)$.
    \end{center}
    The quantified statement is true if $P(x)$ 
    is true for at least one element $x \in S$; while  false 
    if $P(x)$ is false for all $x \in S$.
\end{defi}
\begin{eg}
    The quantified statement $\forall x \in S,\: P(x)$ can be expressed
    \begin{center}
         If $x \in S$, then $P(x)$.
    \end{center}
\end{eg}
\begin{eg}
    Consider $P(x):\: x^2 \geq 0$ over $\mathbb{R}$. Then
    \begin{equation*}
        \forall x \in \mathbb{R},\: x^2 \geq 0
    \end{equation*}
    or equivalently
    \begin{center}
        For every real number $x$, $x^2 \geq 0$.
    \end{center}
    or 
    \begin{center}
        If $x$ is a real number, then $x^2 \geq 0$
    \end{center}
    The statement $ \forall x \in \mathbb{R},\: x^2 \geq 0$ is true
\end{eg}
\begin{eg}
    Consider the open sentence $Q(x):\: x^2 \leq 0$.
    \medbreak
    The statement $\forall x \in \mathbb{R},\: Q(x)$ is false,
     since $Q(1)$ is false. If it were not the case, then there 
     must exist $x$ such that $x^2 > 0$. This negation
     \begin{center}
        There exists a real number $x$ such that $x^2 > 0$
     \end{center}
     can be written
     \begin{equation*}
        \exists x \in \mathbb{R},\: x^2 > 0 \;\text{or}\; 
        \exists x \in \mathbb{R},\: \sim Q(x).
     \end{equation*}
\end{eg}
More generally, if we are considering an open sentence $P(x)$ 
over a domain $S$, then
\begin{equation*}
    \sim (\forall x \in S, P(x)) \equiv \exists x \in S, \sim P(x).
\end{equation*}
\begin{eg}
    Consider $A=\{1,2,3\}$ and $\mathcal{P}(A)$, the power set of $A$.
    \begin{equation*}
         \text{For every set } B \in \mathcal{P}(A),\; A-B \neq \emptyset
    \end{equation*}
    is false since for the subset $B=A=\{1,2,3\}$, $A-B = \emptyset$. \\[1.5ex]
    It can be written as
    \begin{center}
         If $B \subseteq A$, then $A-B \neq \emptyset$ 
    \end{center}
    The negation is 
    \begin{center}
         There exists $B \in \mathcal{P}(A)$ such that $A-B = \emptyset$
    \end{center}
    and can be expressed as
    \begin{center}
         There exist some subset $B$ of $A$ such that $A-B = \emptyset$
    \end{center}
\end{eg}
Generally, if we are considering an open sentence $Q(x)$ over domain $S$.
\begin{equation*}
     \sim (\exists x \in S,\: Q(x))\: \equiv \: \forall x \in S,\: \sim Q(x)
\end{equation*}

For an open sentence containing two variables, the domains of 
the variables need not be the same.
\begin{equation*}
    \sim (\forall s \in S,\: \forall t \in T,\:Q(s,t)) 
    \equiv \exists s \in S,\: \exists t \in T,\: \sim Q(s,t)
\end{equation*}

\subsection{Characterizations}
\begin{defi}[Characterization]
    Suppose that some concept (or object) is expressed in an
    open sentence $P(x)$ over a domain $S$ and $Q(x)$ is 
    another open sentence over the domain $S$ concerning the 
    concept. We say that this concept is \emph{characterized}
    by $Q(x)$, if $\forall x \in S,\: P(x) \Leftrightarrow 
    Q(x)$ is a true statement. The statement $\forall x \in S,\:
    P(x) \Leftrightarrow Q(x)$ is then called a \emph{characterization} 
    of this concept.
\end{defi}
\begin{eg}
    Irrational numbers are defined as real numbers that are not 
    rational and are characterized as real numbers whose decimal
     expansions are nonrepeating.
    
     This provides a characterization of irrational numbers:
     \begin{center}
          \textit{A real number $r$ is irrational if and only if $r$ has 
          a nonrepeating decimal expansion.}
     \end{center}
\end{eg}
\begin{eg}
    We saw that equilateral triangles are defined as triangles 
    whose sides are equal. They are characterized, however, 
    as triangles whose angles are equal.

    Therefore, we have the characterization:
    \begin{center}
         \textit{A triangle $T$ is equilateral if and only if 
         $T$ has three equal angles.}
    \end{center}
\end{eg}

You might think that equilateral triangles are also characterized 
as those triangles having three equal sides but the associated
 biconditional:
 \begin{center}
      \textit{A triangle $T$ is equilateral if and only 
      if $T$ has three equal sides}
 \end{center}
 is not a characterization of equilateral triangles. Indeed, this is 
 the definition we gave of equilateral triangles.
 \begin{eg}
    Define an integer $n$ to be odd if $n$ is not even. Then a 
    characterization of odd integers is
    \begin{center}
         \textit{An integer $n$ is odd if and only if $n^2$ is 
         odd}
    \end{center}
 \end{eg}